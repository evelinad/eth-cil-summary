\chapter{Data Clustering via \(K\)-Means}

Given a set of data points \(\matr{x}_1, \ldots, \matr{x}_N \in \R^D\). The goal is to find a \textit{meaningful partition} of these data points. Therefore, we're interested in labeling each of the data points with a unique label using \(\pi: \{1, \ldots, N\} \to \{1, \ldots, K\}\) or \(\pi: \R^D \to \{1, \ldots, K\}\). The numbering of clusters is arbitrary and the \(k\)-th cluster can be recovered by \(\pi^{-1}(k) \subseteq \{1, \ldots, N\}\) or \(\pi^{-1}(k) \subseteq R^D\).

\section{Vector Quantization}
Partitioning of the space \(\R^D\). Each cluster is represented by a \textit{centroid} \(\matr{u}_k \in \R^D\). Mapping a data point \(\matr{x}\) to a cluster is done via nearest centroid rule, i.e. \(\pi(\matr{x}) = \argmin_{k=1,\ldots,K} \|\matr{u}_k - \matr{x}\|_2\).

\section{Objective Function for \(K\)-Means}
The clustering problem can be formalized as an optimization problem: Find centroids \(\matr{u}_k \in \R^D\) and an assignment function \(\pi\) of data points to clusters, which minimizes a \textit{loss function}/\textit{distortion}, e.g. the squared Euclidean norm.

Useful notation: represent \(\pi\) by a indicator matrix \(\matr{Z}\) defined by
\[
z_{k, n} := \begin{dcases*}
1 & if \(\pi(\matr{x}_n) = k\)\\
0 & otherwise
\end{dcases*}
\]

\begin{definition}[\(K\)-Means Objective Function]
\[
J(\matr{U}, \matr{Z}) = \sum_{n=1}^N \sum_{k=1}^K z_{k, n} \| \matr{x}_n - \matr{u}_k \|_2^2 = \| \matr{X} - \matr{UZ} \|_F^2
\]

with \(\matr{X} = [\matr{x}_1, \ldots, \matr{x}_N] \in \R^{D \times N}\) and \(\matr{U} = [\matr{u}_1, \ldots, \matr{u}_K] \in \R^{D \times K}\)
\end{definition}

\begin{lemma}[\(K\)-Means Objective Function with Euclidean Metric]
\[
J(\matr{U}, \matr{Z}) = \sum_{n=1}^N \sum_{k=1}^K z_{k, n} \| \matr{x}_n - \matr{u}_k \|_F^2 = \| \matr{X} - \matr{UZ} \|_2^2
\]
\end{lemma}

\begin{itemize}
\item Determining optimal \textit{centroids} (\(\matr{U}\)) is easy if the assignments (\(\matr{Z}\)) are given.
\item Determining optimal \textit{assignments} (\(\matr{Z}\)) is easy if the \textit{centroids} (\(\matr{U}\)) are given.
\end{itemize}

\subsection{Optimal Assignment for Given Centroids}
Goal is to compute the optimal assignments (\(\matr{Z}\)), given the centroids (\(\matr{U}\)). Observe that each data point contributes to exactly one term in outer sum. Also note that \(\matr{Z}\) encodes an assignment, i.e. we have \(\sum_{k=1}^K z_{k, n} = 1 \quad \forall n \in \{1, \ldots, N\}\) (each data point is mapped to exactly one cluster).

\begin{enumerate}
\item Minimize each column of \(\matr{Z}\) separately: \[
\matr{z}_{\cdot, n}^\ast = \argmin_{z_{1, n}, \ldots, z_{K, n}} \sum_{k=1}^K z_{k, n} \| \matr{x}_n - \matr{u}_k \|_2^2
\]

\item Optimum is attained by mapping to the closest centroid: \[
z_{k, n}^\ast(\matr{U}) = \begin{dcases*}
1 & if \(k = \argmin_l \| \matr{x}_n - \matr{u}_l \|_2\)\\
0 & otherwise
\end{dcases*}
\]
\end{enumerate}

\subsection{Optimal Centroids for Given Assignment}
Goal is to compute the optimal centroids (\(\matr{U}\)) for a given assignment (\(\matr{Z}\)). The idea is to compute the gradient and set it to zero as a necessary optimality condition).

\begin{enumerate}
\item Look at the (partial) gradient for every centroid \(\matr{u}_k\): \[
\nabla_{\matr{u}_k} J(\matr{U}, \matr{Z}) = \sum_{n=1}^N z_{k, n} \nabla_{\matr{u}_k} \| \matr{x}_n - \matr{u}_k \|_2^2 = -2 \sum_{n=1}^N z_{k, n} (\matr{x}_n - \matr{u}_k)
\]

\item Set the gradient to zero: \[
\nabla_{\matr{U}} J(\matr{U}, \matr{Z}) \overset{!}{=} 0 \Rightarrow \matr{u}_k^\ast(\matr{Z}) = \frac{\sum_{n=1}^N z_{k, n} \matr{x}_n}{\sum_{n=1}^N z_{k,n}}.\quad \text{if } \sum_{n=1}^N z_{k,n} > 0
\]
\end{enumerate}

\subsection{\(K\)-Means Algorithm}
\begin{algorithm}[H]
\caption{\(K\)-Means Algorithm}
\begin{algorithmic}[1]
\Procedure{k-means}{$\matr{X}$, $K \leq N$}
\State initialize $\matr{U}$ with $k$ distinct random data points from $\matr{X}$
\State initialize $\matr{Z} \gets \matr{Z}^\ast(U)$
\Repeat
\State $\matr{U} \gets \matr{U}^\ast(Z)$
\State $\matr{Z}^\text{new} \gets \matr{Z}^\ast(U)$
\State numChanges $\gets \frac{1}{2} \|\matr{Z} - \matr{Z}^\text{new}\|_0$
\State $\matr{Z} \gets \matr{Z}^\text{new}$
\Until{numChanges is $0$}
\Return $(\matr{U}, \matr{Z})$
\EndProcedure
\end{algorithmic}
\end{algorithm}

There are different initialization strategies, here we just use random points.

\subsubsection{Algorithm Analysis}
\begin{itemize}
\item Each iteration costs \(\mathcal{O}(KND)\)

\item \(K\)-means convergence is guaranteed \todo{insert reason from exercise}

\item \(K\)-means optimizes a non-convex objective \todo{insert reason from exercise}. Hence we're not guaranteed to find the global optimum!

\item Finds a local optimum \((\matr{U}, \matr{Z})\) in the following sense:
\begin{itemize}
\item for each \(matr{Z}'\) with \(\frac{1}{2} \| \matr{Z} - \matr{Z}' \|_0 = 1\) (differs in one assignment)
\item \(J(\matr{U}^\ast (\matr{Z}'), \matr{Z}') \geq J(\matr{U}, \matr{Z})\)
\item may gain by changing assignments of \(\geq 2\) points
\end{itemize}

\item \(K\)-means algorithm can be used to compress data:
\begin{itemize}
\item if \(K < N\) with information loss
\item store only centroids and the assignments
\end{itemize}
\end{itemize}

\subsubsection{Problems with Algorithm}
\begin{itemize}
\item Non-convex objective, local minima, sensitive to initialization
\item Not appropriate for non-Euclidean data (need to use other distances)
\item Optimal number of clusters \(K\) is unknown. We have to find a balance between total compression (\(K = 1\)) and no loss of information (\(K = N\)).
\end{itemize}

\section{Choosing the Number of Clusters}
Choosing the number of clusters is also referred to as \textit{model order selection} problem or \textit{model validation}.

\begin{definition}[Clustering Stability]
Repeatedly sample data from the same generation process. ``Good'' algorithms should return clusterings that do not vary much.
\end{definition}

For model selection for \(K\)-means choose the number of clusters \(K\) such that stability is ensured.

\subsection{Testing Clustering Stability}
For a given set of data points and a given number of clusters we can perform the following (high-level) stability test:
\begin{itemize}
\item Generate a perturbed version of the data (e.g. add noise, remove samples)
\item Perform clustering algorithm on all data versions
\item Compute pairwise distance between all clusterings using an appropriate distance measure
\item Compute the instability as the mean distance between all clusterings
\end{itemize}

The above steps should be repeated for different numbers of clusters and we choose the one that minimizes the instability.

The \textit{distance} between two clusterings \(\mathcal{C}, \mathcal{C}'\) of the same data points is defined as follows:
\begin{itemize}
\item Compute the amount of points on which the two clusterings agree or disagree
\item Must be repeated over all cluster labelings as they are arbitrary. This can be done with a permutation function \(\Pi\) 
\end{itemize}

\begin{definition}[Distance of two Clusterings]
Therefore we get the following distance function for two clusterings \(\mathcal{C}, \mathcal{C}'\): \[
d(\mathcal{C}, \mathcal{C}') := \min_{\Pi} \frac{1}{2} \| \matr{Z} - \Pi(\matr{Z}') \|_0
\]

where \(\Pi(\matr{Z}')\) is one of the possible row permutations of \(\matr{Z}'\).
\end{definition}

\todo[inline]{... lecture04 slide 24ff}