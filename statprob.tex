\chapter{Statistics and Probability}
\todo[inline]{Chi-Squared Distribution: lecture02, slide 6}

\todo[inline]{Gamma Distribution: lecture02, slide 7}

\todo[inline]{lecture02, slides 8-12}

\section{Probability}

\begin{definition}[Sample Space]
A \textit{sample space}, denoted by \(\Omega\), is the set of outcomes of a random experiment.
\end{definition}

\begin{definition}[Events]
A subset \(A \subseteq \Omega\) is called an \textit{event}
\end{definition}

\begin{definition}[Probability Distribution]
The function \(p: \Omega \to \R\) is a \textit{probability distribution} if it satisfies the following three axioms:
\begin{enumerate}
\item \(\forall A \subseteq \Omega: p(A) \geq 0\)
\item \(p(\Omega) = 1\)
\item If \(A_1, A_2, \ldots\) are disjoint then \(p(\bigcup_{i=1}^\infty A_i) = \sum_{i=1}^\infty p(A_i)\)
\end{enumerate}
\end{definition}

\begin{notation}
\(p(A) \equiv \operatorname{Pr}[A]\)
\end{notation}

Usually, we do not deal directly with sample spaces. Instead, \textit{random variables} with \textit{probability distributions} are used.

\begin{definition}[Random Variable]
Random variables are a mapping \begin{align*}
X: \Omega &\to \K\\
\omega &\mapsto X(\omega)
\end{align*}
that assigns an element \(X(\omega) \in \K\) to each outcome \(\omega\).
\end{definition}

\begin{notation}[Random Variable]\hfill
\begin{itemize}
\item \(X\) is a random variable
\item \(x\) is a value taken by the random variable \(X\)
\end{itemize}
\end{notation}

If \(\mathcal{X}\) denotes the set of values a random variable \(X\) can take, we can define probabilities directly on \(\mathcal{X}\).

\begin{example}
If the random variable \(X\) denotes the number of heads in two coin tosses, we can set \(\mathcal{X} = \{0, 1, 2\}\). For this example we get:
\begin{align*}
X(HH) &= 2\\
X(TH) &= X(HT) = 1\\
X(TT) &= 0
\end{align*}
and
\begin{align*}
p(X = 0) &:= p(\{TT\})\\
p(X = 1) &:= p(\{HT, TH\})\\
p(X = 2) &:= p(\{HH\}).
\end{align*}
\end{example}

\begin{definition}[Expectation]
For a random variable \(X\) the \textit{expectation} is defined as
\[
\mu_X := E[X] := \sum_{x \in \mathcal{X}} x \cdot P(x).
\]

This can also be defined for a function \(f\) of \(X\):
\[
E[f(X)] := \sum_{x \in \mathcal{X}} f(x) P(x)
\]
\end{definition}

\begin{remark}
The expectation of a random variable is \textit{not} the same as the most likely value \(\max_{x \in \mathcal{X}} P(x)\)
\end{remark}

\begin{definition}[Variance]
For a random variable \(X\) the \textit{variance} is defined by
\[
\operatorname{Var}[X] := E[(X - \mu_X)^2] := \sum_{x \in \mathcal{X}} (x - \mu_X)^2 P(x).
\]

It holds that \(\operatorname{Var}[X] \geq 0\).
\end{definition}

\begin{definition}[Standard Deviation]
For a random variable \(X\) the \textit{standard deviation} is defined as
\[
\sigma_X := \sqrt{\operatorname{Var}[X]}.
\]
\end{definition}

\subsection{Discrete Random Variables}
\begin{definition}[Discrete Random Variable]
\(X\) is a \textit{discrete random variable} iff \(\mathcal{X}\) is a finite or countably infinite set.
\end{definition}

\begin{definition}[Probability Mass Function]
The probability mass function for discrete random variables is \[
P(x) := Pr[X = x]
\]

for which the following must hold:
\begin{itemize}
\item Non-negativity: \(\forall x \in \mathcal{X}: P(x) \geq 0\)
\item Normalization: \(\sum_{x \in \mathcal{X}} P(x) = 1\)
\end{itemize}
\end{definition}

\subsection{Continuous Random Variables}
\begin{definition}[Continuous Random Variables]
\(X\) is a \textit{continuous random variable} iff \(\mathcal{X}\) is an uncountably infinite set.
\end{definition}

The corresponding probability distribution \(p(x)\) is called a \textit{probability density function} and has the following properties:
\begin{itemize}
\item Non-negativity: \(\forall x \in \mathcal{X}: p(x) \geq 0\)
\item Normalization: \(\int_\mathcal{X} p(x) dx = 1\)
\end{itemize}

For continuous random variables: \[p(x) \neq Pr[X = x].\] To get the probability we have to integrate: \[
Pr[a < X < b] = \int_a^b p(x) dx.
\]

\subsection{Joint Distributions}
Let \(X \in \mathcal{X}\) and \(Y \in \mathcal{Y}\) be two random variables.

\begin{definition}[Joint Distribution]
The \textit{joint distribution} of \(X\) and \(Y\) is defined as \[
P(x, y) := Pr[X = x, Y = y]
\]

for which the following must hold:
\begin{itemize}
\item Non-negativity: \(\forall x \in \mathcal{X}, y \in \mathcal{Y}: P(x, y) \geq 0\)
\item Normalization: \(\sum_{x \in \mathcal{X}} \sum_{y \in \mathcal{Y}} P(x, y) = 1\)
\end{itemize}
\end{definition}

\begin{definition}[Marginal Distribution]
\textit{Marginal distribution} for \(X\) is defined as \[
P(x) := Pr[X = x] := \sum_{y \in \mathcal{Y}} P(x, y)
\]
\end{definition}

\begin{definition}[Conditional Distribution]
Knowing that \(Y\) has a known value \(y\) the conditional distribution of \(X\) is defined as \[
P(x|y) := Pr[X = x|Y = y] := \frac{P(x,y)}{P(y)},\quad \text{defined if } P(y) > 0
\]
\end{definition}

\begin{remark}[Chain Rule]
A joint distribution, by definition of conditional distributions, can always be written as a product of conditionals: \[P(x,y) = P(x|y)P(y)\]
\end{remark}

\begin{definition}[Bayes' Rule]
Using the conditional distribution and chain rule we get the \textit{Bayes' Rule}:
\[
P(x|y) = \frac{P(y|x)P(x)}{P(y)}
\]
\end{definition}

\begin{definition}[Independence]
Two random variables \(X\) and \(Y\) are \textit{independent}, if knowing the value of \(X\) does not give any information about the distribution of \(Y\) and vice versa:
\[
P(x|y) = P(x) \Leftrightarrow P(y|x) = P(y)
\]

Equivalently, \(X\) and \(Y\) are independent if their joint distribution factorizes:
\[
P(x,y) = P(x|y)P(y) = P(x)P(y)
\]
\end{definition}

\begin{definition}[Independent and Identically Distributed (IID)]
Random variables \(X_1, X_2, \ldots, X_n\) are independent and identically distributed if
\begin{itemize}
\item each of them has the same (marginal) distribution
\item they are mutually independent
\end{itemize}
\end{definition}

\begin{remark}
If \(X_1, X_2, \ldots, X_n\) are IID, then \[
P(x_1, \ldots, x_n) = P(x_1) \cdots P(x_n) = \prod_{i=1}^n P(x_i)
\]
\end{remark}

\subsection{Multidimensional Moments}
Let \(\matr{X} = [X_1, \ldots, X_n]^\top\) be a vector of random variables.

\begin{definition}[Expectation]
The expectation of \(\matr{X}\) is defined as
\[
E[\matr{X}] := [E[X_1], \ldots, E[X_n]]^\top.
\]
\end{definition}

\begin{definition}[Covariance]
For random variables \(X_i\) and \(X_j\) the covariance is defined as
\[
\operatorname{Cov}[X_i, X_j] := E[(X_i - \mu_{X_i})(X_j - \mu_{X_j})].
\]

And has the following properties:
\begin{itemize}
\item \(\operatorname{Cov}[X_i, X_i] = \operatorname{Var}[X_i]\)
\item If \(X_i, X_j\) are independent: \(\operatorname{Cov}[X_i, X_j] = 0\)
\item \(\operatorname{Cov}[X_i, X_j] > 0\) means roughly that \(X_i\) and \(X_j\) increase and decease together
\item \(\operatorname{Cov}[X_i, X_j] < 0\) roughly means that when \(X_i\) increases \(X_j\) decreases (and vice versa)
\end{itemize}
\end{definition}

\begin{definition}[Covariance Matrix]
For a vector of random variables \(\matr{X} = [X_1, \ldots, X_n]^\top\) the \textit{covariance matrix} is a \(n \times n\) matrix as follows:
\[
\matr{\Sigma}_{\matr{X}} = \begin{bmatrix}
\operatorname{Var}[X_1] & \operatorname{Cov}[X_1, X_2] & \cdots & \operatorname{Cov}[X_1, X_n]\\
\operatorname{Cov}[X_2, X_1] & \operatorname{Var}[X_2] & \cdots & \operatorname{Cov}[X_2, X_n]\\
\vdots & \vdots & \ddots & \vdots\\
\operatorname{Cov}[X_n, X_1] & \operatorname{Cov}[X_n, X_2] & \cdots & \operatorname{Var}[X_n]
\end{bmatrix}
\]

\begin{itemize}
\item The diagonal elements of \(\matr{\Sigma}_{\matr{X}}\) are the variances of each random variable because of \(\operatorname{Cov}[X_i, X_i] = \operatorname{Var}[X_i]\)
\item \(\matr{\Sigma}_{\matr{X}}\) is symmetric because of \(\operatorname{Cov}[X_i, X_j] = \operatorname{Cov}[X_j, X_i]\)
\item \(\matr{\Sigma}_{\matr{X}}\) is positive semi-definite
\end{itemize}
\end{definition}

\subsection{Categorical Distribution}
\begin{itemize}
\item Is a discrete distribution over \(K\) events.
\item Assigns to the \(k\)-th event the probability \(pi_k\).
\item The random variable can be coded as a vector or index
\begin{itemize}
\item \textit{Index coding:} The random variable takes the index of the event that occured
\item \textit{Vector coding:} The random variable is a vector of size \(K\) and if event \(k\) occurs then the \(k\)-th element is set to one, all others to zero.
\end{itemize}
\end{itemize}

\begin{definition}[Probability Mass Function]
The probability mass function is defined by \[
P(\matr{z} | \matr{\pi}) = \prod_{k=1}^K \pi_k^{z_k}
\]
where \(\pi_k\) represents the probability of seeing element \(k\), \(\matr{z}\) is a random variable in vector coding and the following holds:
\begin{itemize}
\item \(\pi_k \geq 0\)
\item \(\sum_{k=1}^K \pi_k = 1\)
\end{itemize}
\end{definition}

\section{Distributions}
\subsection{Gaussian Distribution}
\begin{definition}[Probability Density Function of Gaussian Distribution]
\[
p(x|\mu, \sigma) = \mathcal{N}(x|\mu, \sigma) = \frac{1}{\sqrt{2 \pi} \cdot \sigma} \exp \left\lbrace -\frac{(x - \mu)^2}{2\sigma^2} \right\rbrace
\]

\begin{itemize}
\item \(\mu\) is the mean of the distribution
\item \(\sigma^2\) is the variance of the distribution
\end{itemize}
\end{definition}

\begin{remark}\hfill
\begin{itemize}
\item \(E[X] = \mu\)
\item \(\operatorname{Var}[X] = \sigma^2\)
\end{itemize}
\end{remark}

\begin{definition}[Probability for Gaussian Distribution]
The probability for \(X \in [a, b]\) is given by the integral \[
P(a < X < b) = \int_a^b p(x) dx = \frac{1}{\sqrt{2 \pi} \cdot \sigma} \int_a^b \exp \left\lbrace -\frac{(x - \mu)^2}{2\sigma^2} \right\rbrace dx.
\]
\end{definition}

\subsection{Multivariate Gaussian Distribution}
Generalizes the (univariate) Gaussian distribution to higher dimensions. The sample space is now \(\matr{\mathcal{X}} \subseteq \R^D\). A random vector\footnote{a random variable in vector coding} \(\matr{X} = (X_1, \ldots, X_D)^\top\) has a multivariate normal distribution if every linear combination of its components (i.e. \(Y = a_1 X_1 + \cdots + a_D X_D\)) has a univariate normal distribution.

\begin{remark}
\(E[\matr{X}] = \matr{\mu}\)
\end{remark}

\begin{definition}
\[
p(\matr{x}|\matr{\mu}, \matr{\Sigma}) = \mathcal{N}(\matr{x}|\matr{\mu}, \matr{\Sigma}) :=
\frac{1}{(\sqrt{2\pi})^D |\matr{\Sigma}|^\frac{1}{2}} \exp \left( -\frac{1}{2}(\matr{x} - \matr{\mu}^\top) \matr{\Sigma}^{-1} (\matr{x} - \matr{\mu}) \right)
\]

\begin{itemize}
\item \(\matr{\Sigma}\) is the covariance matrix of \(\matr{X}\)
\item \(|\matr{\Sigma}|\) is the covariance matrix's determinant
\end{itemize}
\end{definition}

\todo{lecture04 slide 38, not clear what this is. Maybe read Wikipedia article regarding this topic?}