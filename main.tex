\documentclass[a4paper,titlepage]{report}

%
% packages
%
\usepackage{
  amssymb,
  amsfonts,
  amsmath,
  mathtools,
  mathtools,
  enumitem,
  wasysym,
  algorithm,
  amsthm,
  color,
}

\usepackage[hidelinks]{hyperref}
\usepackage{algpseudocode}

%
% utf8 fix
%
\usepackage[utf8]{inputenc}

%
% amsthm
%

% fix because we use parskip
\begingroup
    \makeatletter
    \@for\theoremstyle:=definition,remark,plain\do{%
        \expandafter\g@addto@macro\csname th@\theoremstyle\endcsname{%
            \addtolength\thm@preskip\parskip
            }%
        }
\endgroup

% theorem definitions
\theoremstyle{plain}
\newtheorem{lemma}{Lemma}[section]
\newtheorem{theorem}{Theorem}[section]

\theoremstyle{definition}
\newtheorem{definition}{Definition}[section]

\theoremstyle{remark}
\newtheorem{example}{Example}[section]
\newtheorem{remark}{Remark}[section]
\newtheorem{notation}{Notation}[section]

%
% disable paragraph indentation
%
\usepackage[parfill]{parskip}

%
% number sets
%
\newcommand{\R}{\mathbb{R}}
\newcommand{\Z}{\mathbb{Z}}
\newcommand{\N}{\mathbb{N}}
\newcommand{\Q}{\mathbb{Q}}
\newcommand{\C}{\mathbb{C}}
\newcommand{\F}{\mathbb{F}}
\newcommand{\E}{\mathbb{E}}
\newcommand{\K}{\mathbb{K}}
\newcommand{\LL}{\mathcal{L}}
\newcommand{\powerset}{\mathcal P}

%
% notation helpers
%
\newcommand{\matr}[1]{\boldsymbol{\mathrm{#1}}}

%
% Todo/Comments
%
\newcommand{\mynote}[3]{
  \textcolor{#2}{
    \fbox{\bfseries\sffamily\scriptsize#1}
    {\small\textsf{\emph{#3}}}
 }
}

\newcommand\todo[1]{\mynote{G}{red}{#1}}

%
% math symbols/functions
%
\DeclareMathOperator*{\argmin}{arg\,min}
\DeclareMathOperator*{\argmax}{arg\,max}

%
% section settings
%
\setcounter{secnumdepth}{3}

%
% document
%

\begin{document}

% title
\title{Computational Intelligence Lab SS15}
\author{Gregor Wegberg}
\date{\today}
\maketitle

% index
\tableofcontents
\clearpage

% chapters
\chapter{Linear Algebra Basics}
\section{Notation}
Let \(\matr{U}\) be a matrix:
\begin{itemize}
\item \(\matr{u} = (u_1, u_2, \ldots, u_n)^\top\) where \(\matr{u}\) is a vector and \(u_i,\, i \in \{1, \ldots, n\}\) its elements
\item Vectors always considered to be column vectors
\item \(\matr{u}_k\) or \(\matr{u}_{\cdot, k}\) is the \(k\)-th column of \(\matr{U}\)
\item \(\matr{u}^\top_d\) or \(\matr{u}_{d, \cdot}\) is the \(d\)-th row of \(\matr{U}\)
\item \(\matr{u}_{d, k}\) is the element in \(d\)-th row and \(k\)-th column
\item \(\R^{D \times N} \ni \matr{U} = (\matr{u_1}, \matr{u_2}, \ldots, \matr{u_n}) = \begin{pmatrix}
u_{1,1} & \cdots & u_{1, N} \\
\vdots & \ddots & \vdots \\
u_{D, 1} & \cdots & u_{D, N}
\end{pmatrix}\)
\end{itemize}

\section{Scalar Product}
Let \(\matr{x}, \matr{y} \in \R^{D \times 1}\) then \[ \langle \matr{x}, \matr{y} \rangle := \matr{x}^\top \matr{y} = (x_1, x_2, \ldots, x_D) \begin{pmatrix}
y_1 \\ y_2 \\ \vdots \\ y_n
\end{pmatrix} = \sum_{d=1}^D x_d \cdot y_d\]

Basic properties and useful lemmas
\begin{itemize}
\item \(\langle \matr{x}, \matr{x} \rangle = \matr{x}^\top \matr{x}\) is the \textit{squared} Euclidean length/norm of \(\matr{x}\), i.e. \(\langle \matr{x}, \matr{x} \rangle = \| \matr{x} \|^2\)
\item the \(L_2\) Norm (aka Euclidean norm/length): \(\| \matr{x} \|_2 := \sqrt{\langle \matr{x}, \matr{x} \rangle} = \sqrt{\matr{x}^\top \matr{x}}\).
\item The angle between two vectors \(\theta = \angle(\matr{x}, \matr{y})\) is generally known by the equation \[\langle \matr{x}, \matr{y} \rangle = \| \matr{x} \|_2 \| \matr{y} \|_2 \cos(\theta)\]
\item if \(\matr{y}\) is a unit vector (i.e. \(\| \matr{x} \|_2 = 1\)), then \[\langle \matr{x}, \matr{y} \rangle = \| \matr{x} \|_2 \underbrace{\| \matr{y} \|_2}_{= 1} \cos(\theta) = \| \matr{x} \|_2 \cos(\theta)\] and it represents the magnitude of the projection of \(\matr{x}\) onto \(\matr{y}\)
\item \(\langle \matr{x}, \matr{y} \rangle = 0 \Leftrightarrow \matr{x}\) and \(\matr{y}\) are orthogonal.
\end{itemize}

\section{Vector Space}
\begin{itemize}
\item A \textit{vector space} is a set of vectors which are closed under vector addition and (scalar) multiplication.
\item A \textit{subspace} is a subset of the vectors of a vector space which is itself a vector space. Must always include the origin, i.e. the zero vector.
\end{itemize}

\begin{definition}[Column Space]
Is the space defined by the set of all vectors \(\matr{b}\) solving \(\matr{A} \matr{x} = \matr{b}\) for any \(\matr{x}\).

This is also known as the \textit{range}: \[\operatorname{range}(\matr{A}) := \{\matr{b}\ |\ \exists \matr{x}: \matr{Ax} = \matr{b}\}\]
\end{definition}

\begin{definition}[Nullspace]
Is the space defined by the set of all vectors \(\matr{x}\) solving \(\matr{Ax} = \matr{0}\):
\[\operatorname{null}(\matr{A}) := \{\matr{x}\ |\ \matr{Ax} = \matr{0}\}\]
\end{definition}

\begin{definition}[Spanning the Space]
If all vectors of a space can be expressed by a linear combination of the vectors \(\matr{v_1}, \matr{v_2}, \ldots, \matr{v_n}\), then \(\matr{v_1}, \matr{v_2}, \ldots, \matr{v_n}\) \textit{span} the space.
\end{definition}

\begin{definition}[Basis]
A \textit{basis} \(B\) of a vector space \(V\) is a \textbf{linearly independent} subset of \(V\) that \textbf{spans} \(V\).
\end{definition}

Properties of vector space basis:
\begin{itemize}
\item There can be multiple different basis for a space.
\item All basis have the same number of vectors
\item Each basis is a maximal set of linearly independent vectors and minimal set of spanning vectors.
\end{itemize}

\begin{definition}[Dimension]
The \textit{dimension} of a vector space is the number of vectors in a/any basis for the space.
\end{definition}

\begin{definition}[Rank]
The \textit{column rank} of \(\matr{A}\) is the dimension of \(\operatorname{range}(A)\).

This is equivalent to the maximum number of linearly independent column vectors in \(\matr{A}\).

It can be shown that the \textit{row rank} and column rank are always equal.
\end{definition}

\begin{theorem}[Fundamental Theorem of Linear Algebra]
If \(\matr{A}\) is \(M \times N\) with rank \(R\), then
\begin{itemize}
\item \(\operatorname{range}(\matr{A})\) has dimension \(R\)
\item \(\operatorname{null}(\matr{A})\) has dimension \(N - R\)
\item row space of \(\matr{A} = \operatorname{range}(\matr{A}^\top)\) has dimension \(R\)
\end{itemize}
\end{theorem}

\section{Linear Independence}
Vectors \(\matr{v_1}, \matr{v_2}, \ldots, \matr{v_n}\) are \textit{linearly independent} if
\[c_1 \cdot \matr{v_1} + c_2 \cdot \matr{v_2} + \cdots + c_n \cdot \matr{v_n} = 0 \text{ implies } c_1 = c_2 = \cdots = 0\]

Therefore, if we view these vectors \(\matr{v_i}\) as columns of a matrix \(\matr{V}\) the vectors are linearly independent if the null space only contains the origin.

\section{Inverse}
Let \(\matr{A}\) be a square (\(M \times M\)) matrix. The following statements are equivalent:
\begin{itemize}
\item \(\matr{A}\) is invertible/nonsingular
\item \(\matr{A}\) has full rank: \(\operatorname{rank}(\matr{A}) = M\)
\item \(\matr{A}\)'s columns form a basis for the whole space \(\R^M\): \(\operatorname{range}(\matr{A}) = \R^M\)
\item \(\operatorname{null}(\matr{A}) = \{\matr{0}\}\)
\item \(\matr{Ax} = \matr{b}\) has only one solution for each \(\matr{b}\)
\item \(0\) is not an eigenvalue of \(\matr{A}\)
\end{itemize}

Basic properties:
\begin{itemize}
\item \(\matr{A}\matr{A}^{-1} = \matr{A}^{-1}\matr{A} = \matr{I}\)
\item \((k\matr{A})^{-1} = k^{-1}\matr{A}^{-1}\) for nonzero \(k\)
\item \((\matr{A}^\top)^{-1} = (\matr{A}^{-1})^\top\)
\end{itemize}

\section{Eigenvalues and Eigenvectors}
\begin{definition}
Let \(\matr{A}\) be a square matrix. The set of eigen-pairs \((\underbrace{\lambda}_\text{Eigenvalue}, \underbrace{\matr{u}}_\text{Eigenvector})\) is a solution for \(\matr{Au} = \lambda\matr{u}\).

\textbf{Eigenvector} \(\matr{u}\): The direction of \(\matr{u}\) is not changed by transformation \(\matr{A}\). It is only scaled by a factor \(\lambda\), the \textbf{Eigenvalue}.
\end{definition}

\begin{definition}[Eigendecomposition]
The \textit{Eigendecomposition} is defined by \(\matr{A} = \matr{U \Lambda U^{-1}}\), where \(\matr{A}\) is a square matrix and \(\matr{\Lambda}\) is a diagonal matrix with the eigenvalues on the diagonal.
\end{definition}
\chapter{Principle Component Analysis (PCA)}

\section{Intrinsic Dimensionality}
The goal is to find the ``information-preserving''/``interesting'' dimensions. Usually we have a high-dimensional space (let it be of dimension \(M\)) and try to project it to a low-dimensional space (i.e. to a dimension \(N\) such that \(0 \leq N \leq M\)) without loosing ``too much'' of the information (with regard to a loss function). This can be used for example for data compression, feature selection or complexity reduction (``noise removal'').

\section{Iterative View}
PCA can be viewed as an iterative process. First search for a line that is the best approximation of all data points. Therefore we search a line trough the data in the high-dimensional space with the smallest sum of distances (the errors) to each point. This first line is the first principle component. Next search for a second line (the second principle component) with the same property plus being orthogonal to the already existing line. Repeat this till the amount of lines found is equal to the desired dimension of the low-dimensional space.

The found principle components span the new low-dimensional space and the data is transformed from the high-dimensional space into the low-dimensional space.

Minimizing the error, i.e. the distances of the data to each principle component, is formally equivalent to maximizing the variance, i.e. maximizing the difference between the data points after they were projected onto the principle component. Requiring to maximize the variance results in preservation of information.

The dimension of the low-dimensional space can be set by defining a percentage of the total variance that should be represented by the low-dimensional space.

\section{Successive Variance Maximization}
\todo{Unclear what this part of lecture02 should explain/mean}
Let \(\{\matr{x}_n\},\ \matr{x}_n \in \R^D,\ n = 1, \ldots, N\). The goal is now to project the data \(\matr{x}_n\) onto a \(K \leq D\) dimensional space while maximizing the variance of the projected data. The data \(\matr{x}_n\) is represented by a matrix \(\matr{X} = [\matr{x}_1, \matr{x}_2, \ldots, \matr{x}_N] \in \R^{D \times N}\).

\section{Statistics}
\begin{definition}[Empirical Mean of original data]
\[
\bar{\matr{x}} = \frac{1}{N} \sum_{n=1}^{N} \matr{x}_n
\]
\end{definition}

\begin{definition}[Covariance of original data]
\[
\matr{\Sigma} = \frac{1}{N}\sum_{n=1}^{N}(\matr{x}_n - \bar{\matr{x}})(\matr{x}_n - \bar{\matr{x}})^\top
\]
\end{definition}

\section{Derivation of PCA using Eigenvalue Decomposition}
Let \(\matr{u}_1\) be the first principle direction/component and \(\matr{u}_2\) the second one.

\subsection{Statistics of Projected Data}
The \textit{mean} of the projected data is \(\matr{u}_1^\top \bar{\matr{x}}\). The \textit{variance} of the projected data is \begin{align*}
\frac{1}{N}\sum_{n=1}^N \left\lbrace \matr{u}_1^\top \matr{x}_n - \matr{u}_1^\top \bar{\matr{x}} \right\rbrace^2 &= \frac{1}{N} \sum_{n=1}^N \left\lbrace \matr{u}_1^\top (\matr{x}_n - \bar{\matr{x}}) \right\rbrace^2\\
&= \frac{1}{N} \sum_{n=1}^N \matr{u}_1\top (\matr{x}_n - \bar{\matr{x}}) (\matr{x}_n - \bar{\matr{x}})^\top \matr{u}_1\\
&= \matr{u}_1^\top \underbrace{\matr{\Sigma}}_{\mathclap{\text{covariance of original data}}} \matr{u}_1
\end{align*}

\subsection{Maximization Problem}
The next step is now to maximize the variance of the projected data, i.e.
\[
\max_{\matr{u}_1} \matr{u}_1^\top \matr{\Sigma} \matr{u}_1,\ \text{such that } \|\matr{u}_1\|_2 = 1.
\]

This can be achieved by the \textit{Lagrangian} of the maximization problem:
\[
\mathcal{L} = \matr{u}_1^\top \matr{\Sigma} \matr{u}_1 + \lambda_1 (1 - \matr{u}_1^\top \matr{u}_1).
\]

Setting \(\frac{\partial}{\partial\matr{u}_1} \mathcal{L} \overset{!}{=} 0\) results in
\[
\matr{\Sigma} \matr{u}_1 = \lambda_1 \matr{u}_1.
\]

We get the solution by observing that \(\matr{u}_1\) is an \textit{eigenvector} of \(\matr{\Sigma}\) and \(\lambda_1\) its associated \textit{eigenvalue}. \(\lambda_1\) is even the variance of the projected data, i.e. \(\lambda_1 = \matr{u}_1^\top \matr{\Sigma} \matr{u}_1\).

Therefore, maximizing the variance results in choosing the eigenvector with the largest associated eigenvalue (the variance). This results in the \textit{first} \textbf{principle direction}.

\subsection{Second Principal Direction}
This time the variance is maximized along the second principal direction \(\matr{u}_2\), i.e. \(\max_{\matr{u}_2} \matr{u}_2^\top \matr{\Sigma} \matr{u}_2\). At the same time it's required that \(\|\matr{u}_2\|_2 = 1\) and \(\matr{u}_2^\top \matr{u}_1 = 0\) (both principle directions are orthogonal).

Again the Lagrangian
\[
\mathcal{L} = \matr{u}_2^\top \matr{\Sigma} \matr{u}_2 + \lambda_2 (1 - \matr{u}_2^\top \matr{u}_2) + \eta (\matr{u}_2^\top \matr{u}_1)
\]
is used to solve the maximization problem by setting \(\frac{\partial}{\partial \matr{u}_2} \overset{!}{=} 0\) resulting in:
\[
2 \matr{\Sigma} \matr{u}_2 - 2 \lambda_2 \matr{u}_2 + \eta \matr{u}_1 = 0.
\]

Because of orthogonality (\(\langle \matr{u}_2, \matr{u}_1 \rangle = \matr{u}_2^\top \matr{u}_1 = 0\)) leads to \(\eta = 0\) and therefore
\[
\matr{\Sigma} \matr{u}_2 = \lambda_2 \matr{u}_2.
\]

As with the first principle component we can select the second one by choosing the eigenvector from \(\matr{\Sigma}\) with second largest eigenvalue \(\lambda_2\).

\section{Eigenvalue Decomposition}
\begin{definition}
The \textit{eigenvalue decomposition} of the covariance matrix contains all relevant information for PCA. The eigenvalue decomposition is described by \[
\matr{\Sigma} = \matr{U \Lambda U}^\top.
\]
\end{definition}

To project from a \(D\)-dimensional space to a \(K\)-dimensional space, with \(K \leq D\), just choose \(K\) eigenvectors \(\{\matr{u}_1, \ldots, \matr{u}_K\}\) with largest associated eigenvalues \(\{\lambda_1, \ldots, \lambda_K\}\).

\section{Minimum Error Formulation}
Let \(\{\matr{u}_d\},\ \matr{u}_d \in \R^D,\ d = 1, \ldots, D\) be an orthonormal basis and \(\{\matr{x_n}\}\) data points.

The \textit{orthogonal projection} of \(\matr{x}_n\) onto \(\matr{u}_d\) is given by \[
z_{n,d} \cdot \matr{u}_d, \text{ with } z_{n,d} := \matr{x}_n^\top \matr{u}_d \in \R.
\]

Accordingly each \(\matr{x}_n\) can be represented in the \(D\)-dimensional space as \[
\matr{x}_n = \sum_{d=1}^D z_{n,d} \cdot \matr{u}_d = \sum_{d=1}^D (\matr{x}_n^\top \matr{u}_d) \matr{u}_d.
\]

\textit{Restricted representation} using \(K\) (\(K < D\)) basis vectors: \[
\tilde{\matr{x}}_n = \sum_{d=1}^K a_{n, d} \cdot \matr{u}_d + \sum_{d=K+1}^D b_d \cdot \matr{u}_d
\]
where \(b_d\) is independent of the data point \(\matr{x}_n\).

The \textit{average approximation error} \(J\) of the restricted representation is measured by \[
J(\{a_{n,d}\}, \{b_d\}) = \frac{1}{N} \sum_{n=1}^N \|\matr{x}_n - \tilde{\matr{x}}_n\|_2^2.
\]

Minimizing of \(J\) with regard to \(a_{n,d}\) is \(a_{n,d} = \matr{x}_n^\top \matr{u}_d\) and with regard to \(b_d\) it's \(\bar{\matr{x}}^\top \matr{u}_d\). By resubstituting \(a_{n,d}\) and \(b_d\) for the \textit{displacement} we get \[
\matr{x}_n - \tilde{\matr{x}}_n = \sum_{d=K+1}^D \left\lbrace (\matr{x}_n - \bar{\matr{x}})^\top \matr{u}_d \right\rbrace \matr{u}_d.
\]

The \textit{displacement vector} is orthogonal to the principle subspace.

Now we resubstitute the displacement into the error criterion which leads to \[
J = \frac{1}{N}\sum_{n=1}^N \sum_{d=K+1}^D (\matr{x}_n^\top \matr{u}_d - \bar{\matr{x}}^\top \matr{u}_d)^2 = \sum_{d=K+1}^D \matr{u}_d^\top \Sigma \matr{u}_d.
\]

\todo{is J the error? appears so because from K+1 to D we get the dimensions that are removed?!}

\section{Matrix Viewpoint}
Let the data \(\matr{x}_i\) be represented as a matrix \(\matr{X} = [\matr{x}_1, \matr{x}_2, \ldots, \matr{x}_N] \in \R^{D \times N}\). The corresponding mean-centered data is \[\bar{\matr{X}} = \matr{X} - \matr{M}\] where \[\matr{M} = [\underbrace{\bar{\matr{x}}, \bar{\matr{x}}, \ldots, \bar{\matr{x}}}_{\mathclap{N \text{ times}}}] \in R^{D \times N}.\]

Projecting \(\bar{\matr{X}}\) into \(\matr{U}_K = [\matr{u}_1, \matr{u}_2, \ldots, \matr{u}_K]\) (\(\matr{u}_i\) is the eigenvector with the \(i\)-th highest eigenvalue) is represented by \[
\underbrace{\bar{\matr{Z}}}_{\mathclap{K \times N}} = \underbrace{\matr{U_K^\top}}_{\mathclap{K \times D}} \cdot \underbrace{\bar{\matr{X}}}_{\mathclap{D \times N}}.
\]

To approximate \(\bar{\matr{X}}\), we return to the original basis with \[
\tilde{\bar{\matr{X}}} = \matr{U}_K \cdot \tilde{\matr{Z}}_K.
\]

\todo{Unclear what this last part exactly tries to explain...}

\section{Performing PCA}
Let \(\matr{X} \in \R^{D \times N}\) be a given observation. After performing PCA we're interested in \(\tilde{\matr{X}} \in \R^{K \times N}\) with \(K \leq K\) such that \(\tilde{\matr{X}}\) is the low-dimensional representation of \(\matr{X}\).

\subsection{Step 1: Calculate the Empirical Mean}
The empirical mean is calculated by \[
\bar{\matr{x}} = \frac{1}{N} \sum_{n=1}^N \matr{x}_n.
\]

This calculates the observed data's mean in each dimension by summing up all the data for each dimension (vector addition) and dividing it by the amount of measurements \(\N\).

With the empirical mean we can construct \(\matr{M} = [\underbrace{\matr{x}, \matr{x}, \ldots, \matr{x}}_{N \text{ times}}]\).

\subsection{Step 2: Center the Data}
Now we have to center the data (relocate the data around \(\matr{0}\)) by subtracting the empirical mean from each measurement: \[
\bar{\matr{X}} = \matr{X} - \matr{M}
\]

\subsection{Step 3: Computing the Covariance Matrix}
\[
\matr{\Sigma} = \frac{1}{N} \sum_{n=1}^N (\matr{x}_n - \bar{\matr{x}}) (\matr{x}_n - \bar{\matr{x}})^\top = \frac{1}{N} \underbrace{\bar{\matr{X}} \bar{\matr{X}}^\top}_{\mathclap{\text{Scatter Matrix } S}}
\]

\todo{Find answer to question in tutorial02 slide 5}

\subsection{Step 4: Eigenvalue Decomposition}
Now compute the eigenvalue decomposition of the covariance matrix: \[
\matr{\Sigma} = \matr{U \Lambda U}^\top.
\]

After that we sort \(\matr{\Lambda}\), and change \(\matr{U}\) accordingly, such that for \(\matr{\Lambda} = \operatorname{diag}[\lambda_1, \ldots, \lambda_D]\) \(\lambda_1 \geq \lambda_2 \geq \ldots \geq \lambda_D\) holds. Note that the \(i\)-th column of \(\matr{U}\) contains the \(i\)-th eigenvector which corresponds to the \(i\)-th eigenvalue in \(\matr{\Lambda}\).

Because \(\matr{\Sigma}\) is a real and symmetric matrix the eigenvalue decomposition will result in \(\matr{U}\) being an orthogonal matrix with orthonormal entries.

\todo{Find answer to question tutorial02 slide 5}

\subsection{Step 5: Model Selection}
Now we have to pick \(K \leq D\) according to our model. From now on we will only keep the \(K\) largest eigenvalues as those capture the maximal variance of the data.

\subsection{Step 6: Transform the Data onto the new Basis with \(K\) Dimensions}
\[
\R^{K \times N} \ni \bar{\matr{Z}} = \matr{U}_K^\top \bar{\matr{X}}
\]

Here \(\matr{U}_K\) denotes the eigenvector matrix from the eigenvalue decomposition, after sorting the eigenvalues in decreasing order, containing only the first \(K\) eigenvectors.

\subsection{Step 7: Reconstruct to Original Basis}
Now we reconstruct the transformed data by transforming it back to the original basis
\[
	\tilde{\bar{\matr{X}}} = \matr{U}_K \bar{Z}
\]
and reversing the shift introduced by centering the data
\[
	\tilde{\matr{X}} = \tilde{\bar{\matr{X}}} + \matr{M}.
\]
\chapter{Singular Value Decomposition (SVD)}

\begin{definition}
Every rectangle, real or complex matrix \(\matr{A}\) has an SVD decomposition:
\[
\underbrace{\matr{A}}_{M \times N} = \underbrace{\matr{U}}_{M \times M} \underbrace{\matr{D}}_{M \times N} \underbrace{\matr{V}^\top}_{N \times N}
\]

\begin{itemize}
\item \(\matr{U}\) is an orthogonal matrix, i.e. \(\matr{U}^\top \matr{U} = \matr{U} \matr{U}^\top = \matr{I}\).

\item \(\matr{D}\) is a diagonal matrix padded with \(\max(M, N) - \min(M, N)\) zero rows or columns.

\item \(\matr{V}\) is an orthogonal matrix, i.e. \(\matr{V}^\top \matr{V} = \matr{V} \matr{V}^\top = \matr{I}\).

\end{itemize}
\end{definition}

\section{Properties}

\subsection{Singular Values}
\begin{itemize}
\item Elements in the diagonal of \(\matr{D}\) are called the \textit{singular values}. They are denoted by \(\sigma_i := d_{i,i},\ i \in \{1, \ldots, \min(M,N)\}\). Accordingly \(\matr{D} = \operatorname{diag}(\sigma_1, \ldots, \sigma_{\min(M,N)})\).

\item By convention the singular values are ordered in decreasing order, i.e. \(\sigma_1 \geq \sigma_2 \geq \ldots \geq \sigma_{\min(M,N)} \geq 0\).

\item The rank of \(\matr{A}\) corresponds to the number of non-zero singular values.
\end{itemize}

\subsection{Singular Vectors}
\begin{itemize}
\item The first \(\operatorname{rank}(\matr{A})\) columns of \(\matr{U}\) are the \textit{left singular vectors}.
\item The left singular vectors are an orthonormal basis for the column space of \(\matr{A}\).
\item The first \(\operatorname{rank}(\matr{A})\) rows of \(\matr{V}^\top\) are the \textit{right singular vectors}.
\item The right singular vectors are an orthonormal basis for the row space of \(\matr{A}\).
\end{itemize}

\subsection{Algebraic Relationships}
\begin{itemize}
\item By multiplying the SVD equation by \(\matr{V}\) we get \[
\matr{AV} = \matr{UD} \ \Leftrightarrow \ \matr{A} \matr{v}_i = \sigma_i \matr{u}_i \quad \forall i \leq \min(M,N)
\]

\item Similarly \[
\matr{A}^\top \matr{U} = \matr{U} \matr{D}^\top \ \Leftrightarrow \ \matr{A}^\top \matr{u}_i = \sigma_i \matr{v}_i \quad \forall i \leq \min(M, N)
\]

\begin{itemize}
\item in case of \(M = N\) and \(\matr{U} = \matr{V}\), i.e. \(\matr{A}\) is symmetric, we get the eigendecomposition. In this case \(\matr{u}_i = \matr{v}_i\) are eigenvectors.
\end{itemize}
\end{itemize}

\subsection{Compact SVD}
For a rank \(R\) matrix \(\matr{A}\) we're often only interested in the singular vectors corresponding to non-zero singular values. In this case a more compact SVD is possible: \[
\underbrace{\matr{A}}_{M \times N} = \underbrace{\bar{\matr{U}}}_{M \times R} \underbrace{\bar{\matr{D}}}_{M \times R} \underbrace{\bar{\matr{V}}^\top}_{R \times R}.
\]

This representation is exact as excess columns of \(\matr{U}\) and \(\matr{V}\) are multiplied with zeros in \(\matr{D}\).

This representation removes information about the basis fo the null space of \(\matr{A}\).

\section{SVD for Collaborative Filtering}
\begin{definition}[Recommender Systems]
A recommender system analyses  patterns of user's interest in items (movies, books, etc.) to provide personalized recommendations that suit a user's previously observed interest. 
\end{definition}

\begin{definition}[Collaborative Filtering]
Collaborative filtering is the approach to design recommender systems by exploiting the collective data from many users and generalizing across users and items.
\end{definition}

\subsection{Movie Example}
Let \(\matr{A}\) correspond to users ranking of movies. The ranking is from \(0\) (lowest) to \(5\) (highest). Rows represent users and columns movies.

Performing SVD on \(\matr{A}\) results in:
\begin{itemize}
\item \(K\) dimensional latent\footnote{hidden, not visible} concept space
\item \(\matr{U}\): Users-to-concept affinity matrix
\item \(\matr{V}\): Movies-to-concept similarity matrix
\item \(\matr{D}\): Diagonal elements represent the ``expressiveness'' of each concept in the data
\end{itemize}

\section{Singular Values and Matrix Norms}
\subsection{Frobenius Norm}
\begin{definition}[Frobenius Norm]
\[
\| \matr{A} \|_F := \sqrt{\sum_{i=1}^M \sum_{j=1}^N a_{i,j}^2} = \sqrt{\operatorname{trace}(\matr{A}^\top \matr{A})} = \sqrt{\sum_{i=1}^{\min(M,N)} \sigma_i^2}
\]
\end{definition}

The Frobenius norm therefore only depends on the singular values of \(\matr{A}\) which follows from the cyclic nature of \(\operatorname{trace}\), i.e. \(\operatorname{trace}(\matr{XYZ}) = \operatorname{trace}(\matr{ZXY})\):
\begin{align*}
\operatorname{trace}(\matr{A}^\top \matr{A}) &=
\operatorname{trace}(\matr{V} \matr{D}^2 \matr{V}^\top)\\
&= \operatorname{trace}(\matr{D}^2)\\
&= \operatorname{trace}(\operatorname{diag}(\sigma_1^2, \ldots, \sigma_{\min(M, N)}^2))\\
&= \sqrt{\sum_{i=1}^{\min(M,N)} \sigma_i^2}
\end{align*}

\subsection{Induced \(p\)-Norms}
\begin{definition}[Induced \(p\)-Norms]
\[
\| \matr{A} \|_p := \sup \{ \| \matr{Ax} \|_p \ : \ \| \matr{x} \|_p = 1 \}
\]
\end{definition}

\begin{definition}[Matrix \(2\)-Norm]
This norm corresponds to the largest singular value.

\[
\| \matr{A} \|_2 := \sup \{ \| \matr{Ax} \|_2 \ : \ \| \matr{x} \|_2 = 1 \} = \sigma_1
\]
\end{definition}

\section{Eckart-Young Theorem}
The reduced SVD provides an optimal low rank approximation in regard to the Frobenius norm.

Let the SVD of \(\matr{A} \in \R^{M \times N}\) be given by \(\matr{A} = \matr{UDV}^\top\) and  \(K \leq \operatorname{rank}(\matr{A})\). Further we define \(\matr{A}_K\) by \[
\matr{A}_K :- \sum_{i=1}^K \sigma_i \matr{u}_i \matr{v}_i^\top
\]
from which follows that \(\operatorname{rank}(\matr{A}_L) = K\).

Looking at \[
\min_{\operatorname{rank}(\matr{B}) = K} \|\matr{A} - \matr{B}\|_F^2 = \|\matr{A} - \matr{A}_K\|_F^2 = \sum_{k = K+1}^{\operatorname{rank}(\matr{A})} \sigma_k^2
\]
results in the observation that \(\matr{A}_K\) is the closest \(\operatorname{rank} K\) approximation to \(\matr{A}\) in the Forbenius matrix sense.

The matrix \(\matr{A}_K\) is also the optimal approximation in the sense of the Euclidean matrix norm:
\[
\min_{\operatorname{rank}(\matr{B}) = K} \|\matr{A} - \matr{B}\|_2 = \|\matr{A} - \matr{A}_K\|_2 = \sigma_{K + 1}
\]

\section{Eigendecomposition with SVD}
The SVD of \(\matr{A}\) can be used to perform an eigendecomposition of \(\matr{AA}^\top\). In this case the columns of \(\matr{U}\) are the eigenvectors of \(\matr{AA}^\top\) because of 
\[
\matr{AA}^\top = \matr{UDV}^\top \matr{VDU}^\top = \matr{UD}^2\matr{U}^\top.
\]

Similarly the rows of \(\matr{V}^\top\) (or columns of \(\matr{V}\)) are the eigenvectors of \(\matr{A}^\top \matr{A}\) because of
\[
\matr{A}^\top \matr{A} = \matr{VDU}^\top \matr{UDV}^\top = \matr{VD}^2 \matr{V}^\top.
\]

\todo{lecture03 slide 32, ``Do It Yourself SVD'' relevant?}

\section{SVD for Symmetric Matrices}
\begin{theorem}
If \(\matr{S}\) is a real and symmetric matrix, i.e. \(\matr{S} = \matr{S}^\top\), then \(\matr{S} = \matr{UDU}^\top\). Where the columns of \(\matr{U}\) are the eigenvectors of \(\matr{S}\) and \(\matr{D}\) is a diagonal matrix with the eigenvalues of \(\matr{S}\).
\end{theorem}

\begin{proof}
Let \(\matr{U}\) be the matrix of eigenvectors places in the columns, i.e. \(\matr{U} = [\matr{u}_1, \ldots, \matr{u}_N]\), then we can write \(\matr{SU} = \matr{UD}\) or \[
[\matr{Su}_1, \ldots, \matr{Su}_N] = [\sigma_1 \matr{u}_1, \ldots, \sigma_N \matr{u}_N]
\]
by definition of the eigenvectors.

Therefore \(\matr{S} = \matr{UDU}^{-1}\), and since \(\matr{U}\) is orthogonal we get \(\matr{U}^{-1} = \matr{U}^\top\) and therefore \(\matr{S} = \matr{UDU}^\top\).
\end{proof}

\section{Relation Between SVD and PCA}
Let \(\matr{A}\) be a \(M \times N\) matrix.
\begin{itemize}
\item The matrix \(\matr{AA}^\top\) is a symmetric matrix describing the similarity between the \textbf{rows} of the matrix \(\matr{A}\).

The eigenvectors \([\matr{u}_1, \ldots, \matr{u}_n]\) of \(\matr{AA}^\top\)...
\begin{itemize}
\item correspond to the \textit{principle component} in a PCA of \(\matr{A}\). They're equal up to a constant, assuming \(\matr{A}\) is centered.
\item are the \textit{left singular vectors} in the SVD decomposition of \(\matr{A}\).
\end{itemize}

\item Similarly, \(\matr{AA}^\top\) describes the similarity between the \textbf{columns} of the matrix \(\matr{A}\) and the eigenvectors of \(\matr{AA}^\top\) are the \textit{right singular vectors} in the SVD decomposition of \(\matr{A}\).

\item The squared singular values of \(\matr{A}\) are the eigenvalues of \(\matr{AA}^\top\) and \(\matr{A}^\top \matr{A}\).
\end{itemize}

Reasons to use SVD instead of PCA:
\begin{itemize}
\item In case of very high dimensional data the covariance matrix \(\matr{\Sigma}\) will have a squared dimension size.
\item Efficient algorithms available for SVD
\end{itemize}
\chapter{Data Clustering via \(K\)-Means}

Given a set of data points \(\matr{x}_1, \ldots, \matr{x}_N \in \R^D\). The goal is to find a \textit{meaningful partition} of these data points. Therefore, we're interested in labeling each of the data points with a unique label using \(\pi: \{1, \ldots, N\} \to \{1, \ldots, K\}\) or \(\pi: \R^D \to \{1, \ldots, K\}\). The numbering of clusters is arbitrary and the \(k\)-th cluster can be recovered by \(\pi^{-1}(k) \subseteq \{1, \ldots, N\}\) or \(\pi^{-1}(k) \subseteq R^D\).

\section{Vector Quantization}
Partitioning of the space \(\R^D\). Each cluster is represented by a \textit{centroid} \(\matr{u}_k \in \R^D\). Mapping a data point \(\matr{x}\) to a cluster is done via nearest centroid rule, i.e. \(\pi(\matr{x}) = \argmin_{k=1,\ldots,K} \|\matr{u}_k - \matr{x}\|_2\).

\section{Objective Function for \(K\)-Means}
The clustering problem can be formalized as an optimization problem: Find centroids \(\matr{u}_k \in \R^D\) and an assignment function \(\pi\) of data points to clusters, which minimizes a \textit{loss function}/\textit{distortion}, e.g. the squared Euclidean norm.

Useful notation: represent \(\pi\) by a indicator matrix \(\matr{Z}\) defined by
\[
z_{k, n} := \begin{dcases*}
1 & if \(\pi(\matr{x}_n) = k\)\\
0 & otherwise
\end{dcases*}
\]

\begin{definition}[\(K\)-Means Objective Function]
\[
J(\matr{U}, \matr{Z}) = \sum_{n=1}^N \sum_{k=1}^K z_{k, n} \| \matr{x}_n - \matr{u}_k \|_2^2 = \| \matr{X} - \matr{UZ} \|_F^2
\]

with \(\matr{X} = [\matr{x}_1, \ldots, \matr{x}_N] \in \R^{D \times N}\) and \(\matr{U} = [\matr{u}_1, \ldots, \matr{u}_K] \in \R^{D \times K}\)
\end{definition}

\begin{lemma}[\(K\)-Means Objective Function with Euclidean Metric]
\[
J(\matr{U}, \matr{Z}) = \sum_{n=1}^N \sum_{k=1}^K z_{k, n} \| \matr{x}_n - \matr{u}_k \|_F^2 = \| \matr{X} - \matr{UZ} \|_2^2
\]
\end{lemma}

\begin{itemize}
\item Determining optimal \textit{centroids} (\(\matr{U}\)) is easy if the assignments (\(\matr{Z}\)) are given.
\item Determining optimal \textit{assignments} (\(\matr{Z}\)) is easy if the \textit{centroids} (\(\matr{U}\)) are given.
\end{itemize}

\subsection{Optimal Assignment for Given Centroids}
Goal is to compute the optimal assignments (\(\matr{Z}\)), given the centroids (\(\matr{U}\)). Observe that each data point contributes to exactly one term in outer sum. Also note that \(\matr{Z}\) encodes an assignment, i.e. we have \(\sum_{k=1}^K z_{k, n} = 1 \quad \forall n \in \{1, \ldots, N\}\) (each data point is mapped to exactly one cluster).

\begin{enumerate}
\item Minimize each column of \(\matr{Z}\) separately: \[
\matr{z}_{\cdot, n}^\ast = \argmin_{z_{1, n}, \ldots, z_{K, n}} \sum_{k=1}^K z_{k, n} \| \matr{x}_n - \matr{u}_k \|_2^2
\]

\item Optimum is attained by mapping to the closest centroid: \[
z_{k, n}^\ast(\matr{U}) = \begin{dcases*}
1 & if \(k = \argmin_l \| \matr{x}_n - \matr{u}_l \|_2\)\\
0 & otherwise
\end{dcases*}
\]
\end{enumerate}

\subsection{Optimal Centroids for Given Assignment}
Goal is to compute the optimal centroids (\(\matr{U}\)) for a given assignment (\(\matr{Z}\)). The idea is to compute the gradient and set it to zero as a necessary optimality condition).

\begin{enumerate}
\item Look at the (partial) gradient for every centroid \(\matr{u}_k\): \[
\nabla_{\matr{u}_k} J(\matr{U}, \matr{Z}) = \sum_{n=1}^N z_{k, n} \nabla_{\matr{u}_k} \| \matr{x}_n - \matr{u}_k \|_2^2 = -2 \sum_{n=1}^N z_{k, n} (\matr{x}_n - \matr{u}_k)
\]

\item Set the gradient to zero: \[
\nabla_{\matr{U}} J(\matr{U}, \matr{Z}) \overset{!}{=} 0 \Rightarrow \matr{u}_k^\ast(\matr{Z}) = \frac{\sum_{n=1}^N z_{k, n} \matr{x}_n}{\sum_{n=1}^N z_{k,n}}.\quad \text{if } \sum_{n=1}^N z_{k,n} > 0
\]
\end{enumerate}

\subsection{\(K\)-Means Algorithm}
\begin{algorithm}[H]
\caption{\(K\)-Means Algorithm}
\begin{algorithmic}[1]
\Procedure{k-means}{$\matr{X}$, $K \leq N$}
\State initialize $\matr{U}$ with $k$ distinct random data points from $\matr{X}$
\State initialize $\matr{Z} \gets \matr{Z}^\ast(U)$
\Repeat
\State $\matr{U} \gets \matr{U}^\ast(Z)$
\State $\matr{Z}^\text{new} \gets \matr{Z}^\ast(U)$
\State numChanges $\gets \frac{1}{2} \|\matr{Z} - \matr{Z}^\text{new}\|_0$
\State $\matr{Z} \gets \matr{Z}^\text{new}$
\Until{numChanges is $0$}
\Return $(\matr{U}, \matr{Z})$
\EndProcedure
\end{algorithmic}
\end{algorithm}

There are different initialization strategies, here we just use random points.

\subsubsection{Algorithm Analysis}
\begin{itemize}
\item Each iteration costs \(\mathcal{O}(KND)\)

\item \(K\)-means convergence is guaranteed \todo{insert reason from exercise}

\item \(K\)-means optimizes a non-convex objective \todo{insert reason from exercise}. Hence we're not guaranteed to find the global optimum!

\item Finds a local optimum \((\matr{U}, \matr{Z})\) in the following sense:
\begin{itemize}
\item for each \(matr{Z}'\) with \(\frac{1}{2} \| \matr{Z} - \matr{Z}' \|_0 = 1\) (differs in one assignment)
\item \(J(\matr{U}^\ast (\matr{Z}'), \matr{Z}') \geq J(\matr{U}, \matr{Z})\)
\item may gain by changing assignments of \(\geq 2\) points
\end{itemize}

\item \(K\)-means algorithm can be used to compress data:
\begin{itemize}
\item if \(K < N\) with information loss
\item store only centroids and the assignments
\end{itemize}
\end{itemize}

\subsubsection{Problems with Algorithm}
\begin{itemize}
\item Non-convex objective, local minima, sensitive to initialization
\item Not appropriate for non-Euclidean data (need to use other distances)
\item Optimal number of clusters \(K\) is unknown. We have to find a balance between total compression (\(K = 1\)) and no loss of information (\(K = N\)).
\end{itemize}

\section{Choosing the Number of Clusters}
Choosing the number of clusters is also referred to as \textit{model order selection} problem or \textit{model validation}.

\begin{definition}[Clustering Stability]
Repeatedly sample data from the same generation process. ``Good'' algorithms should return clusterings that do not vary much.
\end{definition}

For model selection for \(K\)-means choose the number of clusters \(K\) such that stability is ensured.

\subsection{Testing Clustering Stability}
For a given set of data points and a given number of clusters we can perform the following (high-level) stability test:
\begin{itemize}
\item Generate a perturbed version of the data (e.g. add noise, remove samples)
\item Perform clustering algorithm on all data versions
\item Compute pairwise distance between all clusterings using an appropriate distance measure
\item Compute the instability as the mean distance between all clusterings
\end{itemize}

The above steps should be repeated for different numbers of clusters and we choose the one that minimizes the instability.

The \textit{distance} between two clusterings \(\mathcal{C}, \mathcal{C}'\) of the same data points is defined as follows:
\begin{itemize}
\item Compute the amount of points on which the two clusterings agree or disagree
\item Must be repeated over all cluster labelings as they are arbitrary. This can be done with a permutation function \(\Pi\) 
\end{itemize}

\begin{definition}[Distance of two Clusterings]
Therefore we get the following distance function for two clusterings \(\mathcal{C}, \mathcal{C}'\): \[
d(\mathcal{C}, \mathcal{C}') := \min_{\Pi} \frac{1}{2} \| \matr{Z} - \Pi(\matr{Z}') \|_0
\]

where \(\Pi(\matr{Z}')\) is one of the possible row permutations of \(\matr{Z}'\).
\end{definition}

\subsection{Extended Version of Stability Calculations}
\todo[inline]{search difference to the ``not extended'' version and why we care}

Let \(\matr{X}, \matr{X}'\) be two arbitrary data sets of size \(N\) respectively \(N'\).
\begin{enumerate}
\item Cluster \(\matr{X}, \matr{X}'\) to get \(\matr{Z}, \matr{Z}'\)
\item Train a multi-class classifier \(\phi\) on \((\matr{X}, \matr{Z})\)
\item Now we apply \(\phi\) on \(\matr{X}'\) and compare, after permuting, the output with \(\matr{Z}'\): \[
r := \frac{1}{N'} \min_{\Pi \in \mathbb{S}_K} \left\lbrace \sum_{n=1}^{N'} \mathbb{I}_{\left\lbrace \Pi(\phi(\matr{x}'_n)) \neq \matr{z}'_n \right\rbrace} \right\rbrace.
\]
Where \(\mathbb{I}_{\left\lbrace p \right\rbrace}\) is the indicator function and is \(1\) iff the predicate \(p\) is true, \(0\) otherwise. \(\mathbb{S}_K\) is the symmetric group and minimizing over \(\Pi \in \mathbb{S}_K\) compensates for the permutation of the cluster numbers. \todo{symmetric group? not sure what \(\mathbb{S}_K\) really represents}
\end{enumerate}

\subsubsection{Normalization of Stability}
Higher number of clusters makes it more difficult to have a small rate \(r\) of inconsistent cluster assignments. We can define the rate \(r_\text{random} := \frac{K-1}{K}\) for a random assignment given \(K\) clusters. Now we can compare hypotheses with different \(K\) by relating \(r\) to \(r_\text{random}\) as follows.

\begin{definition}[Stability]
\[
\operatorname{stab} := 1 - \frac{r}{r_\text{rand}}
\]

If \(\operatorname{stab} = 1\) then there is no inconsistent assignments, this is the (perfect) goal. If \(\operatorname{stab} = 0\) then our algorithm is not better than random assignment.
\end{definition}

\todo[inline]{not clear about motivation, content and if at all important: lecture04 slide 27ff (``K-means as Matrix Factorization'')}


\chapter{Mixture Models and the Expectation Maximization Algorithm}

\section{Probabilistic Clustering}
Instead of a ``hard'' assignment of data points to a specific cluster (e.g. \(K\)-means algorithm) we now do a probabilistic assignment. Therefore every data point \(\matr{x}_i\) is assigned to each cluster \(k\) with a probability \(z_{k, i}\). Therefore our assignment matrix \(\matr{Z}\) is now constrained by
\begin{itemize}
\item \(\forall k, i: z_{k, i} \in [0, 1]\)
\item \(\sum_{k=1}^K z_{k, i} = 1 \quad \forall n\)
\end{itemize}

As a consequence the objective \(J\) can no longer be interpreted as an approximation matrix factorization problem!

\subsection{Generative Model}
\todo[inline]{still not 100\% sure I got the idea. \url{http://cs229.stanford.edu/materials.html} may be a source for better understanding}
A generative model models how data is generated. For a given data it checks which of the known generation models most likely could generate the given data.

Such a model is parametrized by a set of probability distributions \(\mathcal{P} = \{p_\theta : \theta \in \Theta\}\) with
\begin{itemize}
\item \(Pr_\theta[\matr{x} \in A] = \int_A p_\theta(\matr{x}) d\matr{x}\) for measurable \(A\)
\item \(\Theta\) is the space of possible parameter values (e.g. \(\subseteq \R^M\))
\end{itemize}

Now the question remains how to select the optimal parameter \(\Theta\).

\begin{definition}[Likelihood Function]
Probability of observed data under \(\Theta\) is turned into a \textit{likelihood function} for \(\Theta\) given an outcome.

\begin{itemize}
\item Probability = parameter given, events (data) to be generated
\item Likelihood = outcome (data) given, parameter to be inferred
\end{itemize}

\[
\mathcal{L}(\theta;\matr{X}) := p_\theta(\matr{X}) \overset{\ast}{=} \prod_{n=1}^N p_\theta (\matr{x}_n) \quad \ast \text{IID sampling}
\]
\end{definition}

\begin{definition}[Maximum Likelihood Estimator (MLE)]
\[
\hat{\theta} = \argmax_{\theta \in \Theta} p_\theta (\matr{X}) \overset{\ast}{=} \argmax_{\theta \in \Theta} \sum_{n=1}^N \ln p_\theta (\matr{x}_n)
 \quad \ast \text{IID sampling}
\]
\end{definition}

\section{Mixture Models}
\begin{definition}
For a finite mixture model we define
\[
p_\theta = \sum_{k=1}^K \pi_k p(\matr{x}; \theta_k),\quad \theta = (\pi, \theta_1, \ldots, \theta_K) \in \R^{K + K \cdot M}.
\]

\begin{itemize}
\item mixing proportions: \(\pi \geq 0,\ \sum_{k=1}^K \pi_k = 1\)
\item mixture components: \(p(\cdot;\theta_k)\) with \(\theta_k \in \R^M\)
\end{itemize}
\end{definition}

\subsection{Gaussian Mixture Model}
\begin{definition}
For the \textit{Gaussian Mixture Model (GMM)} the probability distribution is defined as
\[
p_\theta(\matr{x}) = \sum_{k=1}^K \pi_k \ \mathcal{N}(\matr{x}; \matr{\mu}_k, \matr{\Sigma}_k).
\]
\end{definition}

\todo[inline]{completely lost any form of understanding... lecture05 slide 9ff}
\chapter{Non-Negative Matrix Factorization}

\section{Motivation \& Example: Semantic Document Representation}
Given some corpus of text documents (e.g. web pages, PDFs) we're interested in finding a low-dimensional representation of each document and to find \textit{semantic dimensions} (topics, concepts) in the set of documents.

This would allow to...
\begin{itemize}
\item get a more robust representation than just a word-based one
\item define meaningful similarities between documents
\item compress the document corpus or index
\end{itemize}

\begin{definition}[Vocabulary]
The vocabulary is every ``meaningful'' word in a language. This excludes for example all stop words (``the'', ``is'', ``at'', ``which'', ...).

After that we filter out infrequent words, misspellings, tokenizer errors, etc. and do \textit{stemming} (optionally).

\(D\) is the size of the vocabulary.
\end{definition}

\begin{remark}[Stemming]
Stemming is the process of reducing morphological variations of words. For example ``argue'', ``argued'', ``argues'', ... is reduced to the stem ``arg''
\end{remark}

A document is from now on represented by a \textit{bag of words}, i.e. words from the vocabulary ignoring the order of them in the document. This way each document can be represented by a vector of length \(D\) with frequencies/counts of the different words found. This vector is very sparse!

\begin{definition}[Document-Term Matrix]
This matrix \(\matr{X} \in \R_{\geq 0}^{D \times N}\) stores the word counts for each document:
\[
\matr{X} = [\matr{x}_1, \matr{x}_2, \ldots, \matr{x}_N].
\]

Here \(N\) is the number of documents and \(x_{d,n}\) represents the frequency of the \(d\)-th word in the \(n\)-th document.
\end{definition}

\section{Non-Negative Matrix Factorization (NMF)}

\begin{definition}[Non-Negative Matrix Factorization (NMF)]
The NMF of \(\matr{X}\) is
\[
\matr{X} \approx \matr{UZ}
\]
with \(\matr{U} \in \R_{\geq 0}^{D \times K}\) and \(\matr{Z} \in \R_{\geq 0}^{K \times N}\) where
\begin{itemize}
\item \(N\) is the number of documents
\item \(D\) the vocabulary size
\item \(K\) number of dimensions (model design choice)
\item it results in data reduction as \((D+N)K \ll DN\)
\end{itemize}
\end{definition}

\begin{notation}
\(X \in \R_{\geq 0}^{N \times M}\) denotes a \(N \times M\) matrix that has only non-negative entries.
\end{notation}

\section{Classic Latent Semantic Indexing (LSI)}
Uses a truncated SVD:
\[
\tilde{\matr{X}}_K = \matr{U} \tilde{\matr{\Sigma}}_K \matr{V}^\top \approx \matr{X}.
\]

Here \(\tilde{\matr{\Sigma}}_K\) contains only the largest \(K\) singular values and the rest set to zero.

\((\matr{U} \tilde{\matr{\Sigma}}_K)^\top\) is interpreted as a mapping from the \(D\)-dimensional ``word space'' to a \(K\)-dimensional latent ``topic space''.

A new document/query \(\matr{x}\) is mapped by: \( \matr{x} \mapsto (\matr{U} \tilde{\matr{\Sigma}}_K)^\top \matr{x}\).

Already mapped vectors \(\bar{\matr{x}}\) can be compared by the inner product:
\[
\left\langle \bar{\matr{x}}_1, \bar{\matr{x}}_2 \right\rangle = \left(
(\matr{U} \tilde{\matr{\Sigma}}_K)^\top \matr{x}_1
\right)^\top (\matr{U} \tilde{\matr{\Sigma}}_K)^\top \matr{x}_2 = \matr{x}_1^\top \matr{U} \tilde{\matr{\Sigma}}_K^2 \matr{U}^\top \matr{x}_2
\]

\todo[inline]{stopped at lecture06 slide 7}

\chapter{Optimization}
The general optimization problem is to minimize a function \(f(\matr{x})\) with \(\matr{x} \in \R^D\). In this lecture the function is assumed to be \(f: \R^D \to \R\), continuous and differentiable. Therefore, we are interested in finding \(\matr{x}^\star \in \R^D\) such that \(f(\matr{x}^\star)\) is minimal. 

\section{Coordinate Descent}
The basic idea is to change only one variable while keeping all others fixed.

\begin{algorithm}[H]
\caption{Coordinate Descent}
\begin{algorithmic}[1]
\Procedure{coordinate\_descent}{}
\State initialize \(\matr{x}^{(0)} \in \R^D\)
\For{t = 0:maxIter}
	\State \(d \gets\) uniformly at random from \(1, \ldots, D\)
	\State \(u^\star \gets \argmin_{u \in \R} f(x_1^{(t)}, x_2^{(t)}, \ldots, x_{d-1}^{(t)}, u, x_{d+1}^{(t)}, \ldots, x_D^{(t)})\)
	\State \(\matr{x}_d^{(t+1)} \gets u^\star\) \Comment{use the newly found value for the \(d\)-th variable}
	\State \(\matr{x}_{d'}^{(t+1)} \gets \matr{x}_{d'}^{(t)} \; \forall d' \neq d\) \Comment{use all other variables as they were in the previous round}
\EndFor \Comment{the last ``version'' (\(\matr{x}^{(t+1)}\)) should lead to the smallest value of \(f\) we were able to find in maxIter iterations}
\EndProcedure
\end{algorithmic}
\end{algorithm}

\section{Gradient Descent}
A function (note the kind of functions we consider in the beginning of this chapter) \emph{decreases} in value fastest if one follows in \emph{opposite} direction of the gradient of that function at the current point. Going into the same direction as the gradient would lead us to a (local) maximum.

The \emph{gradient} of a function \(f: \R^D \to \R\) is
\[
\nabla f(\matr{x}) := \left( \frac{\partial f(\matr{x})}{\partial \matr{x}_1}, \ldots, \frac{\partial f(\matr{x})}{\partial \matr{x}_D} \right)^\top \in \R^D
\]

\begin{algorithm}[H]
\caption{Gradient Descent}
\begin{algorithmic}[1]
\Procedure{gradient\_descent}{}
\State initialize \(\matr{x}^{(0)}\)
\For{t = 0:maxIters}
	\State \(\matr{x}^{(t+1)} \gets \matr{x}^{(t)} - \gamma \nabla f(\matr{x}^{(t)})\)
\EndFor
\EndProcedure
\end{algorithmic}
\end{algorithm}

The \emph{stepsize} \(\gamma\) is usually decreased in each iteration such that \(\gamma \approx \frac{1}{t}\).

\section{Stochastic Gradient Descent}
For function of the form
\[
f(\matr{x}) = \frac{1}{N} \sum_{n=1}^N f_n(\matr{x})
\]
the stochastic gradient descent can be applied.

\begin{algorithm}[H]
\caption{Stochastic Gradient Descent}
\begin{algorithmic}[1]
\Procedure{stochastic\_gradient\_descent}{}
\State initialize \(\matr{x}^{(0)} \in \R^D\)
\For{t = 0:maxIter}
	\State \(n \gets\) uniformly at random from \(1, \ldots, N\)
	\State \(\matr{x}^{(t+1)} \gets \matr{x}^{(t)} - \gamma \nabla f_n(\matr{x}^{(t)})\)
\EndFor
\EndProcedure
\end{algorithmic}
\end{algorithm}

Compared to the previous gradient descent this method is computationally cheaper, while keeping the estimate of the gradient unbiased (\( E[\nabla f_n(\matr{x})] = \nabla f(\matr{x}) \)) as long as we select \(n\) at random.

Here too the stepsize is usually decreased, i.e. \(\gamma \approx \frac{1}{t}\).

\section{Constrained Optimization}
In case of constrained optimization we are interested in minimizing \(f(\matr{x})\) as before, but for a \(\matr{x} \in Q \subseteq \R^D\), i.e. \(\matr{x}\) is expected to be part of a given set.

\subsection{Projected Gradient Descent}
Use gradient descent, but project the resulting \(\matr{x}^{(i)}\) into \(Q\). To project \(\matr{x}^{(i)}\) into \(Q\) the projection function
\[
P_Q(\matr{x}) := \argmin_{y \in Q} \|y - x\|
\]
is used.

\begin{algorithm}[H]
\caption{Projected Gradient Descent}
\begin{algorithmic}[1]
\Procedure{projected\_gradient\_descent}{}
\State initialize \(\matr{x}^{(0)}\)
\For{t = 0:maxIters}
	\State \(\matr{x}^{(t+1)} \gets P_Q\left(\matr{x}^{(t)} - \gamma \nabla f(\matr{x}^{(t)})\right)\)
\EndFor
\EndProcedure
\end{algorithmic}
\end{algorithm}

\subsection{Constrained Optimization as Unconstrained}
A constrained optimization can be modified to be an unconstrained one using an \emph{penalty function}. Let \(P(\matr{x})\) be a penalty function. Now we can use unconstrained optimization algorithms to solve \(f(\matr{x})\) by optimizing \(f(\matr{x}) + P(\matr{x})\) instead.

There are multiple ways to define meaningful penalty function:
\begin{itemize}
\item Indicator function: \( P(\matr{x}) = I_Q(\matr{x}) := \begin{cases} 0 & x \in Q\\ \infty & x \neq Q \end{cases}\)
\item Penalize error, e.g. \(Q = \{\matr{x} \in \R^D \ |\ \matr{Ax} = \matr{b}\} \Rightarrow P(\matr{x}) = \lambda \|\matr{Ax} - \matr{b}\|^2\)
\item Linearized penalty functions
\end{itemize}

\section{Duality for Constrained Optimization}
We are interested in minimizing \(f(\matr{x})\) with subject to
\begin{align*}
g_i(\matr{x}) \leq 0,\, i = 1, \ldots, m\\
h_i(\matr{x}) = 0,\, i = 1, \ldots, p
\end{align*}
with \(f(\matr{x})\) being the \emph{objective function}, \(g_i(\matr{x})\) the \emph{inequality constraint functions}, and \(h_i(\matr{x})\) the \emph{affine equality constraint functions}. Further, \[h_i(\matr{x}) = \matr{a}_i^\top \matr{x} - b_i.\]

This constrained minimization problem can be translated into a unconstrained one:
\begin{eqnarray*}
&\min f(\matr{x}) + \sum_{i=1}^m I\_(g_i(\matr{x})) + \sum_{i=1}^p I_0(h_i(\matr{x}))\\
&\text{with}\\
&I\_(y) := \begin{cases} 0 & y \leq 0\\ \infty & y > 0 \end{cases} \quad\text{and}\quad
I_0(y) := \begin{cases} 0 & u = 0\\ \infty & u \neq 0 \end{cases}
\end{eqnarray*}

It is possible \todo{why?} to approximate \(I\_(y)\) and \(I_0(y)\) linearly with \(\lambda_i y\) and \(\nu_i y\) leading us to the \emph{Lagrangian}
\[
L(\matr{x}, \matr{\lambda}, \matr{\nu}) := f(\matr{x}) + \sum_{i = 1}^m \matr{\lambda}_i g_i(\matr{x}) + \sum_{i = 1}^p \matr{\nu}_i h_i(\matr{x}).
\]
\(\lambda_i\) and \(\nu_i\) are called \emph{Langrange multipliers}. The \emph{Lagrange dual function} is defined by
\[
d(\matr{\lambda}, \matr{\nu}) := \inf_{x} L(\matr{x}, \matr{\lambda}, \matr{\nu}) \; \in \R.
\]

Since \(\lambda_i u \leq I\_(u)\) and \(\nu_i u \leq I_0(u)\) for all \(u\) the dual function gives us always a lower bound on the primal value \(f(\matr{x})\) of any feasible \(\matr{x}\).

\begin{definition}[Lagrange dual problem]
\todo{I don't completely get this... try to understand it}
The Lagrange dual problem is the problem of maximizing the Lagrange dual function in regard to \(\matr{\lambda} \geq 0\). This gives us a lower bound on the unknown solution value of \(f(\matr{x}^\star)\).

In case the original optimization problem is \emph{convex}, and some additional conditions \todo{... way to go to with all this details in the slides ...}, the solution value of the dual problem is \emph{equal} to the solution value of \(f(\matr{x}^\star)\) of the primal problem!
\end{definition}

\section{Convex Optimization}
\begin{definition}[Convex Set]
A set \(Q\) is \emph{convex} if the line segment between two points \(\matr{x}, \matr{y} \in Q\) lies in \(Q\), i.e. \(\theta \matr{x} + (1-\theta) \matr{y} \in Q\).

Further:
\begin{itemize}
\item Intersection of convex sets are convex too
\item Projections onto convex sets are unique, and often efficient to compute (the \(P_Q\) function already discussed above)
\end{itemize}
\end{definition}

\begin{definition}[Convex Function]
A function \(f: \R^D \to \R\) is convex if
\begin{itemize}
\item \(\operatorname{dom}(f)\) is a convex set (this is the domain of the set, i.e. all ``valid input variables'' of the function), and
\item \(\forall \matr{x}, \matr{y} \in \operatorname{dom}(f), 0 \leq \theta \leq 1: f(\theta \matr{x} + (1-\theta) \matr{y}) \leq \theta f(\matr{x}) + (1-\theta) f(\matr{y})\), i.e. the line segment is ``above'' the graph of \(f\).
\end{itemize}
\end{definition}

\begin{definition}[Graph of a function]
The graph of a function \(f\) is defined by
\[
\{(\matr{x}, f(\matr{x}))\ |\ \matr{x} \in \operatorname{dom}(f)\}.
\]
\end{definition}

\begin{definition}[Epigraph of a function]
The epigraph of a function \(f\) is defined by
\[
\{(\matr{x}, t)\ |\ \matr{x} \in \operatorname{dom}(f), f(\matr{x}) \leq t\}.
\]

A function \(f\) is convex iff its epigraph is a convex set.
\end{definition}

Examples for convex functions:
\begin{itemize}
\item Linear functions: \(f(\matr{x}) = \matr{a}^\top \matr{x}\)
\item Affine functions : \(f(\matr{x}) = \matr{a}^\top \matr{x} + b\)
\item Exponential: \(f(\matr{x}) = e^{\alpha \matr{x}}\)
\item Every norm on \(\R^D\)
\end{itemize}

\todo{Exam: Somehow I feel like they might ask for a proof that norms are convex, as is shown on slide 36 in lecture07}

\emph{Convex optimization problems} are optimizations of a convex function \(f\) such that \(\matr{x}^\star \in Q\), with \(Q\) being a convex set. Note: \(\R^D\) is convex! All such convex optimization problems have the property that every local minimum is also a global minimum. All the previously shown algorithms in this chapter can be used for convex optimization problems and will \emph{converge} to the global optimum.

\section{SubGradient Descent}
\todo{not sure how/why it works}

 

\chapter{Sparse Coding}
Signals can be represented in many different ways (e.g. Fourier series). Natural signals can often be represented by a sparse representation.

\section{Signal Compression}
Let \(\matr{x}\) be a signal and \(\matr{U}\) an \(L \times L\) \emph{orthogonal} matrix. Using this orthogonal matrix we can a change of basis through the transformation of the signal into \(\matr{z} = \matr{Ux}\) while \emph{preserving the energy}:
\[
\| \matr{z} \|^2 = \| \matr{Ux} \|^2 = \left\langle
	\sum_{l} \matr{x}_l \matr{U}_{:, l},
	\sum_{l} \matr{x}_l \matr{U}_{:, l}
\right\rangle
\overbrace{=}^{\substack{\text{due to} \\ \text{orthogonality}}}
\sum_{l=1}^{L} \matr{x}_l^2 \underbrace{\langle \matr{U}_{:, l}, \matr{U}_{:, l}}_{= 1?} \rangle
= \| x \|^2
\]
\todo{Not sure the last step is really valid for ``just'' an orthogonal matrix \(\matr{U}\). Should be only possible for ortho\emph{normal} \(\matr{U}\)?}

Using the transformed signal \(\matr{z}\) we can now only keep the \(K\) ``strongest'' values. If we chose \(K \ll L\) we get the compressed signal \(\hat{\matr{z}}\).

The signal can be reconstructed using the inverse transformation, i.e. \(\hat{\matr{x}} = \matr{U^\top \hat{z}}\). Note this is efficient to compute due to orthogonality: \(\matr{U}^{-1} = \matr{U}^\top\).

\section{Decomposition and Reconstruction}
Let \(\matr{x}\) be a signal vector of length \(L\), \(\{\matr{v}_1, \matr{v}_2, \ldots, \matr{v}_L\}\) an orthonormal basis, and the coefficients representing \(\matr{x}\) in this orthonormal basis are \(z_l = \langle \matr{x}, \matr{v}_l \rangle\).

Using the coefficients and the basis we can reconstruct the original signal with
\[
\matr{x} = \sum_{l=1}^L z_l \matr{v}_l = \sum_{l=1}^L \langle \matr{x}, \matr{v}_l \rangle \matr{v}_l.
\]

By keeping only a subset of \(K\) basis functions/vectors (represented by the set \(\sigma\)) we get a sparse representation \(\hat{\matr{x}}\) of the original signal \(\matr{x}\):
\[
\hat{\matr{x}} = \sum_{k \in \sigma} z_k \matr{v}_k
\]
The resulting \emph{reconstruction error} is
\[
\| \matr{x} - \hat{\matr{x}} \|^2 = \sum_{k \not\in \sigma} |\langle \matr{x}, \matr{v}_k \rangle|^2
\]
since \(\langle \matr{v}_k, \matr{v}_l \rangle = 0\) if \(k \neq l\).

\section{Discrete Fourier Transform}
\todo{Some pictures, no explanation. Yay!}

\section{Wavelet Basis}
\todo{Less pictures, same amount of explanation.}

\section{Compressive Sensing}
The basic idea is to compress data while gathering it, instead of collecting large amounts of data to only throw most of it away for compression. Compressing while gathering also usually leads to faster acquisition time, smaller power consumption, and less storage space required.

The assumption is that the original signal \(\matr{x} \in \R^D\) is sparse in some orthonormal basis \(\matr{U}\). In other words, the sparse representation \(\matr{z}\) of \(\matr{x}\) (\(\matr{x} = \matr{Uz}\)) only keeps \(K\) largest coefficients, i.e. \(\| z \|_0 = K\) where \(\| \cdot \|_0\) is the number of non-zero elements.

\todo{not sure how the first part of slide 29 in lecture08 is connected to the second part, and by it the connection to the following slides}

\todo{lecture09, lecture10 skipped due to not clearly understand lecture08. Must revisit}
\chapter{Robust PCA}
\section{Introduction}
The goal is to find a low-rank representation of a matrix \(\matr{X}\) corrupted by a sparse perturbation:
\[
\matr{X} \approx \underbrace{\matr{L_0}}_{\mathclap{\text{low-rank representation of } \matr{X}}} + \overbrace{\matr{S_0}}^{\mathclap{\text{sparse perturbation matrix}}}.
\]
This is a so called \emph{additive decomposition}, i.e. we decompose the matrix into a sum of matrices.

Such a decomposition of \(\matr{X}\) with the aim to get a low-rank representation of it and a sparse matrix can be formulated as
\begin{eqnarray*}
\text{minimize}_{\matr{L}, \matr{S}}& \quad\quad &\operatorname{rank}(\matr{L}) + \lambda \cdot \operatorname{card}(\matr{S})\\
\text{subject to}& &\matr{L} + \matr{S} = \matr{X}
\end{eqnarray*}
with \(\operatorname{card}(\cdot)\) counting the number of non-zero entries. However, this approach is not feasible in general as it is not a convex optimization problem.

An alternative approach is to do a \emph{convex relaxation}
\begin{eqnarray*}
\text{minimize}_{\matr{L}, \matr{S}}& \quad\quad &\|\matr{L}\|_\star + \lambda \|\matr{S}\|_1\\
\text{subject to}& &\matr{L} + \matr{S} = \matr{X}.
\end{eqnarray*}
Here \(\|\cdot\|_\star\) is the \emph{nuclear norm} defined by \(\sum_{i=1}^{\min(m,n)} \sigma_i\) with \(\sigma_i\) being the \(i\)-th singular value of the \(n \times m\) matrix. \(\| \cdot \|_1\) denotes the sum of absolute values of all elements of the matrix.

This alternative approach is not the same problem as our first optimization formularization. However, the alternative allows to efficiently solve the optimization, and under the right conditions will even lead to a solution which is also the solution for the first optimization problem. \todo{not clear why and how, but well\ldots}

\todo{\ldots not sure how relevant slides 6 - 23 (lecture11) are and how to summarize/process them. skipped.}

\section{Classical PCA Shortcomings}
Classical PCA is very sensitive to outliers. One single data point outside the ``usual'' range leads to a complete change of the principal component. PCA is said to have a \emph{breakpoint} of zero. The \emph{breakpoint} is defined as the smallest proportion of data elements which can be changed without resulting in an arbitrarily-large change in the estimator.

\section{Robust PCA}
The basic idea is to use the previously mentioned additive decomposition to extract the noisy outliers. This outliers would end up in \(\matr{S}_0\) leaving us with \(\matr{L}_0\) to perform classical PCA on.

The \emph{Principle Component Pursuit (PCP)} problem, previously encountered as \emph{convex relaxation} (\todo{what is the connection, is it really the same and just two names?}), is
\begin{eqnarray*}
\text{minimize}_{\matr{L}, \matr{S}}& \quad\quad &\|\matr{L}\|_\star + \lambda \|\matr{S}\|_1\\
\text{subject to}& &\matr{L} + \matr{S} = \matr{X}.
\end{eqnarray*}
and, under some conditions, leads to the optimal solution of the additive decomposition into a low-rank matrix and a sparse perturbation matrix.

The conditions are as follows:
\begin{itemize}
	\item \(\matr{X}\) can not be low-rank \textbf{and} sparse
	\item \todo{\ldots and they lost me again \ldots}
\end{itemize}


\chapter{Statistics and Probability}
\todo[inline]{Chi-Squared Distribution: lecture02, slide 6}

\todo[inline]{Gamma Distribution: lecture02, slide 7}

\todo[inline]{lecture02, slides 8-12}

\section{Probability}

\begin{definition}[Sample Space]
A \textit{sample space}, denoted by \(\Omega\), is the set of outcomes of a random experiment.
\end{definition}

\begin{definition}[Events]
A subset \(A \subseteq \Omega\) is called an \textit{event}
\end{definition}

\begin{definition}[Probability Distribution]
The function \(p: \Omega \to \R\) is a \textit{probability distribution} if it satisfies the following three axioms:
\begin{enumerate}
\item \(\forall A \subseteq \Omega: p(A) \geq 0\)
\item \(p(\Omega) = 1\)
\item If \(A_1, A_2, \ldots\) are disjoint then \(p(\bigcup_{i=1}^\infty A_i) = \sum_{i=1}^\infty p(A_i)\)
\end{enumerate}
\end{definition}

\begin{notation}
\(p(A) \equiv \operatorname{Pr}[A]\)
\end{notation}

Usually, we do not deal directly with sample spaces. Instead, \textit{random variables} with \textit{probability distributions} are used.

\begin{definition}[Random Variable]
Random variables are a mapping \begin{align*}
X: \Omega &\to \K\\
\omega &\mapsto X(\omega)
\end{align*}
that assigns an element \(X(\omega) \in \K\) to each outcome \(\omega\).
\end{definition}

\begin{notation}[Random Variable]\hfill
\begin{itemize}
\item \(X\) is a random variable
\item \(x\) is a value taken by the random variable \(X\)
\end{itemize}
\end{notation}

If \(\mathcal{X}\) denotes the set of values a random variable \(X\) can take, we can define probabilities directly on \(\mathcal{X}\).

\begin{example}
If the random variable \(X\) denotes the number of heads in two coin tosses, we can set \(\mathcal{X} = \{0, 1, 2\}\). For this example we get:
\begin{align*}
X(HH) &= 2\\
X(TH) &= X(HT) = 1\\
X(TT) &= 0
\end{align*}
and
\begin{align*}
p(X = 0) &:= p(\{TT\})\\
p(X = 1) &:= p(\{HT, TH\})\\
p(X = 2) &:= p(\{HH\}).
\end{align*}
\end{example}

\begin{definition}[Expectation]
For a random variable \(X\) the \textit{expectation} is defined as
\[
\mu_X := E[X] := \sum_{x \in \mathcal{X}} x \cdot P(x).
\]

This can also be defined for a function \(f\) of \(X\):
\[
E[f(X)] := \sum_{x \in \mathcal{X}} f(x) P(x)
\]
\end{definition}

\begin{remark}
The expectation of a random variable is \textit{not} the same as the most likely value \(\max_{x \in \mathcal{X}} P(x)\)
\end{remark}

\begin{definition}[Variance]
For a random variable \(X\) the \textit{variance} is defined by
\[
\operatorname{Var}[X] := E[(X - \mu_X)^2] := \sum_{x \in \mathcal{X}} (x - \mu_X)^2 P(x).
\]

It holds that \(\operatorname{Var}[X] \geq 0\).
\end{definition}

\begin{definition}[Standard Deviation]
For a random variable \(X\) the \textit{standard deviation} is defined as
\[
\sigma_X := \sqrt{\operatorname{Var}[X]}.
\]
\end{definition}

\subsection{Discrete Random Variables}
\begin{definition}[Discrete Random Variable]
\(X\) is a \textit{discrete random variable} iff \(\mathcal{X}\) is a finite or countably infinite set.
\end{definition}

\begin{definition}[Probability Mass Function]
The probability mass function for discrete random variables is \[
P(x) := Pr[X = x]
\]

for which the following must hold:
\begin{itemize}
\item Non-negativity: \(\forall x \in \mathcal{X}: P(x) \geq 0\)
\item Normalization: \(\sum_{x \in \mathcal{X}} P(x) = 1\)
\end{itemize}
\end{definition}

\subsection{Continuous Random Variables}
\begin{definition}[Continuous Random Variables]
\(X\) is a \textit{continuous random variable} iff \(\mathcal{X}\) is an uncountably infinite set.
\end{definition}

The corresponding probability distribution \(p(x)\) is called a \textit{probability density function} and has the following properties:
\begin{itemize}
\item Non-negativity: \(\forall x \in \mathcal{X}: p(x) \geq 0\)
\item Normalization: \(\int_\mathcal{X} p(x) dx = 1\)
\end{itemize}

For continuous random variables: \[p(x) \neq Pr[X = x].\] To get the probability we have to integrate: \[
Pr[a < X < b] = \int_a^b p(x) dx.
\]

\subsection{Joint Distributions}
Let \(X \in \mathcal{X}\) and \(Y \in \mathcal{Y}\) be two random variables.

\begin{definition}[Joint Distribution]
The \textit{joint distribution} of \(X\) and \(Y\) is defined as \[
P(x, y) := Pr[X = x, Y = y]
\]

for which the following must hold:
\begin{itemize}
\item Non-negativity: \(\forall x \in \mathcal{X}, y \in \mathcal{Y}: P(x, y) \geq 0\)
\item Normalization: \(\sum_{x \in \mathcal{X}} \sum_{y \in \mathcal{Y}} P(x, y) = 1\)
\end{itemize}
\end{definition}

\begin{definition}[Marginal Distribution]
\textit{Marginal distribution} for \(X\) is defined as \[
P(x) := Pr[X = x] := \sum_{y \in \mathcal{Y}} P(x, y)
\]
\end{definition}

\begin{definition}[Conditional Distribution]
Knowing that \(Y\) has a known value \(y\) the conditional distribution of \(X\) is defined as \[
P(x|y) := Pr[X = x|Y = y] := \frac{P(x,y)}{P(y)},\quad \text{defined if } P(y) > 0
\]
\end{definition}

\begin{remark}[Chain Rule]
A joint distribution, by definition of conditional distributions, can always be written as a product of conditionals: \[P(x,y) = P(x|y)P(y)\]
\end{remark}

\begin{definition}[Bayes' Rule]
Using the conditional distribution and chain rule we get the \textit{Bayes' Rule}:
\[
P(x|y) = \frac{P(y|x)P(x)}{P(y)}
\]
\end{definition}

\begin{definition}[Independence]
Two random variables \(X\) and \(Y\) are \textit{independent}, if knowing the value of \(X\) does not give any information about the distribution of \(Y\) and vice versa:
\[
P(x|y) = P(x) \Leftrightarrow P(y|x) = P(y)
\]

Equivalently, \(X\) and \(Y\) are independent if their joint distribution factorizes:
\[
P(x,y) = P(x|y)P(y) = P(x)P(y)
\]
\end{definition}

\begin{definition}[Independent and Identically Distributed (IID)]
Random variables \(X_1, X_2, \ldots, X_n\) are independent and identically distributed if
\begin{itemize}
\item each of them has the same (marginal) distribution
\item they are mutually independent
\end{itemize}
\end{definition}

\begin{remark}
If \(X_1, X_2, \ldots, X_n\) are IID, then \[
P(x_1, \ldots, x_n) = P(x_1) \cdots P(x_n) = \prod_{i=1}^n P(x_i)
\]
\end{remark}

\subsection{Multidimensional Moments}
Let \(\matr{X} = [X_1, \ldots, X_n]^\top\) be a vector of random variables.

\begin{definition}[Expectation]
The expectation of \(\matr{X}\) is defined as
\[
E[\matr{X}] := [E[X_1], \ldots, E[X_n]]^\top.
\]
\end{definition}

\begin{definition}[Covariance]
For random variables \(X_i\) and \(X_j\) the covariance is defined as
\[
\operatorname{Cov}[X_i, X_j] := E[(X_i - \mu_{X_i})(X_j - \mu_{X_j})].
\]

And has the following properties:
\begin{itemize}
\item \(\operatorname{Cov}[X_i, X_i] = \operatorname{Var}[X_i]\)
\item If \(X_i, X_j\) are independent: \(\operatorname{Cov}[X_i, X_j] = 0\)
\item \(\operatorname{Cov}[X_i, X_j] > 0\) means roughly that \(X_i\) and \(X_j\) increase and decease together
\item \(\operatorname{Cov}[X_i, X_j] < 0\) roughly means that when \(X_i\) increases \(X_j\) decreases (and vice versa)
\end{itemize}
\end{definition}

\begin{definition}[Covariance Matrix]
For a vector of random variables \(\matr{X} = [X_1, \ldots, X_n]^\top\) the \textit{covariance matrix} is a \(n \times n\) matrix as follows:
\[
\matr{\Sigma}_{\matr{X}} = \begin{bmatrix}
\operatorname{Var}[X_1] & \operatorname{Cov}[X_1, X_2] & \cdots & \operatorname{Cov}[X_1, X_n]\\
\operatorname{Cov}[X_2, X_1] & \operatorname{Var}[X_2] & \cdots & \operatorname{Cov}[X_2, X_n]\\
\vdots & \vdots & \ddots & \vdots\\
\operatorname{Cov}[X_n, X_1] & \operatorname{Cov}[X_n, X_2] & \cdots & \operatorname{Var}[X_n]
\end{bmatrix}
\]

\begin{itemize}
\item The diagonal elements of \(\matr{\Sigma}_{\matr{X}}\) are the variances of each random variable because of \(\operatorname{Cov}[X_i, X_i] = \operatorname{Var}[X_i]\)
\item \(\matr{\Sigma}_{\matr{X}}\) is symmetric because of \(\operatorname{Cov}[X_i, X_j] = \operatorname{Cov}[X_j, X_i]\)
\item \(\matr{\Sigma}_{\matr{X}}\) is positive semi-definite
\end{itemize}
\end{definition}

\subsection{Categorical Distribution}
\begin{itemize}
\item Is a discrete distribution over \(K\) events.
\item Assigns to the \(k\)-th event the probability \(pi_k\).
\item The random variable can be coded as a vector or index
\begin{itemize}
\item \textit{Index coding:} The random variable takes the index of the event that occured
\item \textit{Vector coding:} The random variable is a vector of size \(K\) and if event \(k\) occurs then the \(k\)-th element is set to one, all others to zero.
\end{itemize}
\end{itemize}

\begin{definition}[Probability Mass Function]
The probability mass function is defined by \[
P(\matr{z} | \matr{\pi}) = \prod_{k=1}^K \pi_k^{z_k}
\]
where \(\pi_k\) represents the probability of seeing element \(k\), \(\matr{z}\) is a random variable in vector coding and the following holds:
\begin{itemize}
\item \(\pi_k \geq 0\)
\item \(\sum_{k=1}^K \pi_k = 1\)
\end{itemize}
\end{definition}

\section{Distributions}
\subsection{Gaussian Distribution}
\begin{definition}[Probability Density Function of Gaussian Distribution]
\[
p(x|\mu, \sigma) = \mathcal{N}(x|\mu, \sigma) = \frac{1}{\sqrt{2 \pi} \cdot \sigma} \exp \left\lbrace -\frac{(x - \mu)^2}{2\sigma^2} \right\rbrace
\]

\begin{itemize}
\item \(\mu\) is the mean of the distribution
\item \(\sigma^2\) is the variance of the distribution
\end{itemize}
\end{definition}

\begin{remark}\hfill
\begin{itemize}
\item \(E[X] = \mu\)
\item \(\operatorname{Var}[X] = \sigma^2\)
\end{itemize}
\end{remark}

\begin{definition}[Probability for Gaussian Distribution]
The probability for \(X \in [a, b]\) is given by the integral \[
P(a < X < b) = \int_a^b p(x) dx = \frac{1}{\sqrt{2 \pi} \cdot \sigma} \int_a^b \exp \left\lbrace -\frac{(x - \mu)^2}{2\sigma^2} \right\rbrace dx.
\]
\end{definition}

\subsection{Multivariate Gaussian Distribution}
Generalizes the (univariate) Gaussian distribution to higher dimensions. The sample space is now \(\matr{\mathcal{X}} \subseteq \R^D\). A random vector\footnote{a random variable in vector coding} \(\matr{X} = (X_1, \ldots, X_D)^\top\) has a multivariate normal distribution if every linear combination of its components (i.e. \(Y = a_1 X_1 + \cdots + a_D X_D\)) has a univariate normal distribution.

\begin{remark}
\(E[\matr{X}] = \matr{\mu}\)
\end{remark}

\begin{definition}
\[
p(\matr{x}|\matr{\mu}, \matr{\Sigma}) = \mathcal{N}(\matr{x}|\matr{\mu}, \matr{\Sigma}) :=
\frac{1}{(\sqrt{2\pi})^D |\matr{\Sigma}|^\frac{1}{2}} \exp \left( -\frac{1}{2}(\matr{x} - \matr{\mu}^\top) \matr{\Sigma}^{-1} (\matr{x} - \matr{\mu}) \right)
\]

\begin{itemize}
\item \(\matr{\Sigma}\) is the covariance matrix of \(\matr{X}\)
\item \(|\matr{\Sigma}|\) is the covariance matrix's determinant
\end{itemize}
\end{definition}

\todo{lecture04 slide 38, not clear what this is. Maybe read Wikipedia article regarding this topic?}

\end{document}
